\documentclass[device=normal, lang=en]{elegantbook}
\usepackage{amsmath}
\usepackage{amssymb}
\usepackage{float}
\usepackage{extarrows}
\usepackage{physics}

\newcommand{\e}{\mathrm{e}}
\newcommand{\E}{\mathbf{E}}
\newcommand{\cov}{\mathrm{cov}}
\renewcommand{\var}{\mathrm{var}}


\definecolor{pgcolor}{RGB}{251, 250, 248}
\pagecolor{pgcolor}
\numberwithin{equation}{section}

% \theoremstyle{definition} %
% \newtheorem{property}{Property}[section] %

\title{Probability Notes}
\author{FHYQ-Dong}
\date{\today}
\version{8.0}
\cover{images/Note-cover.png}
\definecolor{customcoverlinecolor}{RGB}{82, 59, 148}
\colorlet{coverlinecolor}{customcoverlinecolor}


\begin{document}

\maketitle
\frontmatter

\tableofcontents
\mainmatter


\input{chaps/chap_Probability_Space.tex}
\chapter{Conditional Probability}
\section{Conditional Probability}
\begin{definition}[Conditional Probability]
    Conditional probability is the probability of an event $A$ given that another event $B$ has already occurred. It is denoted by $P(A|B)$ and is defined as:
    \begin{equation}
        \mathbf{P}(A|B) = \frac{\mathbf{P}(A \cap B)}{\mathbf{P}(B)}
    \end{equation}
    Note that when $\mathbf{P}(B) = 0$, the conditional probability is undefined.
\end{definition}

We can easily check the conditional probability satisfies the properties of probability measure, which means it is a legitimate probability on a new universe.
\begin{itemize}
    \item Non-negativity: $\mathbf{P}(A|B) \geq 0$.
    \item Normalization: $\mathbf{P}(\varOmega|B) = 1$.
    \item Additivity: when $A$ and $B$ are disjoint, $\mathbf{P}(A \cup B|C) = \frac{\mathbf{P}((A \cup B) \cap C)}{\mathbf{P}(C)} = \frac{\mathbf{P}(A \cap C) + \mathbf{P}(B \cap C)}{\mathbf{P}(C)} = \mathbf{P}(A|C) + \mathbf{P}(B|C)$. (The second equality is due to $A \cap C$ and $B \cap C$ are disjoint).
\end{itemize}

\begin{example}[Discrete and Continuous]
    For discrete case, the conditional probability can be calulated by:
    \begin{equation}
        \mathbf{P}(A|B) = \frac{\#~of~elements~in~A}{\#~of~elements~in~B}
    \end{equation}
    For continuous case, the conditional probability can be calculated by:
    \begin{equation}
        \mathbf{P}(A|B) = \frac{area~of~A}{area~of~B}
    \end{equation}
\end{example}

When solving a problem, the following equations may help:
\begin{align}
    \mathbf{P}(A \cap D) &= \mathbf{P}(A) \cdot \mathbf{P}(D|A) = \mathbf{P}(D) \cdot \mathbf{P}(A|D) \\
    \mathbf{P}(A \cap B \cap C) &= \mathbf{P}(A) \cdot \mathbf{P}(B|A) \cdot \mathbf{P}(C|A \cap B)
\end{align}
For the second equation: 1.event $A$ occurs; 2.event $B$ occurs given $A$; 3.event $C$ occurs given $A$ and $B$. Or
\begin{align*}
    \mathbf{P}(A \cap B \cap C) &= \mathbf{P}(A \cap B) \cdot \mathbf{P}(C | A \cap B) \\ 
    &= \mathbf{P}(A) \cdot \mathbf{P}(B | A) \cdot \mathbf{P}(C | A \cap B)
\end{align*}
\begin{remark}
    These equations are not fixed, depending on $\mathbf{P}(A)$, $\mathbf{P(A \cap B)}$ and $\mathbf{P}(A | B)$ which are easier to calculate.
\end{remark}

\section{Total Probability Theorem}
We can calculate the total probability with \textit{Devide and Conquer} strategy:
\begin{enumerate}
    \item Devide event $B$ by $A_1, A_2, \cdots A_n$, note that $\bigcup_{i=1}^n A_i = \varOmega$ and $A_i \cap A_j = \emptyset$.
    \item Calculate $\mathbf{P}(B | A_i)$ which are easier to calculate.
    \item Sum all the probabilities.
\end{enumerate}
\begin{figure}[H]
    \centering
    \includegraphics[width=0.3\textwidth]{images/Note-2.1.png}
    \caption{Total Probability Theorem}
    \label{fig:total-probability}
\end{figure}

\begin{theorem}[Total Probability Theorem]
    Let $A_1, A_2, \cdots, A_n$ be a partition of the sample space $\varOmega$. Then for any event $B$,
    \begin{equation}
        \mathbf{P}(B) = \sum_{i=1}^{n} \mathbf{P}(A_i) \cdot \mathbf{P}(B | A_i)
    \end{equation}
\end{theorem}
\begin{example}[Die Rolling]
    You roll a fair four-sided die. If the result is 1 or 2, you roll once more but otherwise, you stop. What is the probability that the sum total of rolls is at least 4?
    \begin{solution}
        \begin{enumerate}
            \item Let $A_i = \{the~first~roll~is~i\}, B = \{the~sum~total~of~rolls~is~at~least~4\}$.
            \item $\mathbf{P}(A_1) = \mathbf{P}(A_2) = \mathbf{P}(A_3) = \mathbf{P}(A_4) = \frac{1}{4}$.
            \item $\mathbf{P}(B | A_1) = \frac{1}{2}, \mathbf{P}(B | A_2) = \frac{3}{4}, \mathbf{P}(B | A_3) = 0, \mathbf{P}(B | A_4) = 1$.
            \item $\mathbf{P}(B) = \frac{9}{16}$.
        \end{enumerate}
    \end{solution}
\end{example}


\section{Bayes' Theorem}
If we know:
\begin{itemize}
    \item ``Prior'' probabilities $\mathbf{P}(A_i)$.
    \item ``Likelihood'' probabilities $\mathbf{P}(B | A_i)$.
\end{itemize}
We wish we can calculate the ``Posterior'' probabilities $\mathbf{P}(A_i | B)$. 

\begin{theorem}[Bayes' Theorem]
    Let $A_1, A_2, \cdots, A_n$ be a partition of the sample space $\varOmega$. Then for any event $B$
    \begin{equation}
    \begin{aligned}
        \mathbf{P}(A_i | B) &= \frac{\mathbf{P}(A_i \cap B)}{\mathbf{P}(B)} \\
        &= \frac{\mathbf{P}(A_i) \cdot \mathbf{P}(B | A_i)}{\sum_{j=1}^{n} \mathbf{P}(A_j) \cdot \mathbf{P}(B | A_j)} \\
        &= \frac{\mathbf{P}(A_i) \cdot \mathbf{P}(B | A_i)}{\sum_{j=1}^{n} \mathbf{P}(A_j) \cdot \mathbf{P}(B | A_j)}
    \end{aligned}
    \end{equation}
\end{theorem}

A: cause, B: result. Bayes' Theorem is used to calculate the cause given the result. \textit{(inference based on probability)}

\input{chaps/chap_Independence.tex}
\input{chaps/chap_Discrete_Random_Variables.tex}
\chapter{Multiple Random Variables}
\section{Conditioning}
\begin{definition}[Conditional PMF]
    The conditional PMF of $X$ given $A$ is defined as
    \begin{align}
        p_{X|A}(x) = \mathbf{P}(X = x | A) = \frac{\mathbf{P}(\{X = x\} \cap A)}{\mathbf{P}(A)}
    \end{align}
    Specifically, if given $Y = y$, then
    \begin{align}
        p_{X|Y = y}(x) = \mathbf{P}(X = x | Y = y) = \frac{\mathbf{P}(\{X = x\} \cap \{Y = y\})}{\mathbf{P}(Y = y)}
    \end{align}
\end{definition}
Since $\mathbf{P}(A) = \sum_{x} \mathbf{P}(\{X = x\} \cap A)$, we have $\sum_{x} p_{X|A}(x) = 1$.
The conditional PMF limits the sample space to $A$.
\begin{definition}[Conditional Expectation]
    The conditional expectation of $X$ given $A$ is defined as
    \begin{align}
        \mathbf{E}[X|A] = \sum_{x} x p_{X|A}(x)
    \end{align}
    Specifically, if given $Y = y$, then
    \begin{align}
        \mathbf{E}[X|Y = y] = \sum_{x} x p_{X|Y}(x|y)
    \end{align}
\end{definition}
\begin{theorem}[Total Expectation Theorem]
    Partition the sample space into disjoint events $A_{1}, A_{2}, \cdots$, then
    \begin{equation}
    \begin{aligned}
        \mathbf{P}(B) &= \sum_{i} \mathbf{P}(A_{i}) \mathbf{P}(B|A_{i}) \\ 
        p_{X}(x) &= \sum_{i} \mathbf{P}(A_{i}) p_{X|A_{i}}(x) \\
        \mathbf{E}[X] &= \sum_{i} \mathbf{P}(A_{i}) \mathbf{E}[X|A_{i}]
    \end{aligned}
    \end{equation}
    From line 2 to line 3, we multiply $x$ on both sides and sum over $x$.
    \begin{figure}[H]
        \centering
        \includegraphics[width=0.3\textwidth]{images/Note-5.1.png}
        \caption{Total Expectation Theorem}
    \end{figure}
\end{theorem}
\begin{example}[Geometric PMF 1]
    Toss a fair coin independently until a head occurs. Let $X$ be the number of tosses. What is the PMF, expectation, and variance of $X$? 
    \begin{solution}
        \begin{equation}
        \begin{aligned}
            \mathbf{E}[X] &= \sum_{k=1}^{\infty} k \mathbf{P}(X = k) = \sum_{k=1}^{\infty} k (1-p)^{k-1} p = \frac{1}{p} \\ 
            \text{var}(X) &= \mathbf{E}[X^{2}] - \mathbf{E}[X]^{2} = \frac{1-p}{p^{2}}
        \end{aligned}
        \end{equation}
    \end{solution}
\end{example}
\begin{property}[Memoryless Property]
    Given that $X > 2$, the random variable $X - 2$ has same geometric PMF with $X$ (not given that $X > 2$). 
    \begin{equation}
    \begin{aligned}
        p_{X|X > 2}(x) &= \frac{\mathbf{P}(\{X > 2\} \cap \{X = x\})}{\mathbf{P}(X > 2)} \\ 
        &= \frac{(1 - p)^{k-1}p}{1 - p - p(1-p)} = (1-p)^{k-3}p
    \end{aligned}
    \end{equation}
    If we \textit{shift} $k$ by 2, then the PMF is the same. That is to say, the random variable $X - 2$ given $X > 2$ has the same geometric PMF with $X$, and thus $\mathbf{E}[(X - 2)|X > 2] = \mathbf{E}[X]$.
\end{property}
\begin{example}[Geometric PMF 2]
    Toss a fair coin independently until a head occurs. Let $X$ be the number of tosses. What is the PMF, expectation, and variance of $X$? 
    \begin{solution}
        If we use the memoryless property, then 
        \begin{equation}
        \begin{aligned}
            \mathbf{E}[X] &= \mathbf{P}(X = 1)\mathbf{E}[X | X = 1] + \mathbf{P}(X > 1)\mathbf{E}[X | X > 1] \\ 
            &= p + (1-p)\mathbf{E}[(X - 1 + 1) | X > 1] \\
            &= p + (1-p)(1 + \mathbf{E}[X]) \\ 
            &\Rightarrow \mathbf{E}[X] = \frac{1}{p}
        \end{aligned}
        \end{equation}
        Similarly
        \begin{equation}
        \begin{aligned}
            \mathbf{E}[X^2] &= \mathbf{P}(X = 1)\mathbf{E}[X^2 | X = 1] + \mathbf{P}(X > 1)\mathbf{E}[X^2 | X > 1] \\
            &= p + (1-p)\mathbf{E}[(X - 1 + 1)^2 | X > 1] \\
            &= p + (1-p)(1 + 2\mathbf{E}[X] + \mathbf{E}[X^2]) \\
            &\Rightarrow \mathbf{E}[X^2] = \frac{2-p}{p^2}
        \end{aligned}
        \end{equation}
        Then
        \begin{equation}
            \text{var}(X) = \mathbf{E}[X^2] - \mathbf{E}[X]^2 = \frac{1-p}{p^2}
        \end{equation}
    \end{solution}
\end{example}

\section{Multiple Discrete Random Variables}
\begin{definition}[Joint PMF]
    The joint PMF of $X$ and $Y$ is defined as
    \begin{align}
        p_{X, Y}(x, y) = \mathbf{P}(\{X(\omega) = x\} \cap \{Y(\omega) = y\})
    \end{align}
\end{definition}
\begin{remark}
    \textbf{All the RVs are defined on the same probability space $(\Omega, \mathcal{F}, \mathbf{P})$. For those RVs defined on different probability spaces, we need to define a new probability space $\Omega = \Omega_X \times \Omega_Y$ previously.}
\end{remark}
\begin{property}[Properties of Joint PMF] ~
    \begin{itemize}
        \item Added up to 1: $\sum_x \sum_y p_{X, Y}(x, y) = 1$.
        \item Marginal PMF: $p_{X}(x) = \sum_y p_{X, Y}(x, y)$, $p_{Y}(y) = \sum_x p_{X, Y}(x, y)$. (Total Probability Theorem)
        \item Conditional PMF: $p_{X|Y}(x|y) = \frac{p_{X, Y}(x, y)}{p_{Y}(y)}$, $p_{Y|X}(y|x) = \frac{p_{X, Y}(x, y)}{p_{X}(x)}$.
    \end{itemize}
\end{property}
If we have a RV function of multiple RVs: $Z = g(X, Y)$, then the PMF of $Z$ is
\begin{align}
    p_{Z}(z) = \sum_{x, y: g(x, y) = z} p_{X, Y}(x, y)
\end{align}
\begin{definition}[Expectation of Multiple Random Variables]
    Recall that 
    \begin{align*}
        \mathbf{E}[g(X)] = \sum_{x} g(x) p_{X}(x)
    \end{align*}
    Then the expectation of $g(X, Y)$ is
    \begin{align}
        \mathbf{E}[g(X, Y)] = \sum_{x, y} g(x, y) p_{X, Y}(x, y)
    \end{align}
\end{definition}
\begin{property}[Properties of Expectation] ~
    \begin{itemize}
        \item In general, $\mathbf{E}[X + Y] \neq \mathbf{E}[X] + \mathbf{E}[Y]$.
        \item $\mathbf{E}[\alpha X + \beta Y + \gamma] = \alpha \mathbf{E}[X] + \beta \mathbf{E}[Y] + \gamma$. \textbf{This property holds for any RVs, despite of the independence.} (go back to the definition of expectation and you will see the linearity, which is also the reason of property 1.)
    \end{itemize}
\end{property}
\begin{example}[Coin tossing]
    Toss a coin independently. The coin may not be fair. 
    \begin{solution}
        \begin{equation}
            X_i = \begin{cases}
                1, & \text{if the $i$-th toss is head} \\ 
                0, & \text{otherwise}
            \end{cases}
        \end{equation}
        The sum $X = \sum_{i=1}^n X_i$. Then 
        \begin{equation}
            \mathbf{E}[X] = \sum_{i=1}^n \mathbf{E}[X_i] = np
        \end{equation}
        \begin{equation}
        \begin{aligned}
            \text{var}(X) &= \mathbf{E}[X^2] - \mathbf{E}[X]^2 = (\sum_{i=1}^{N} X_i^2 + \sum_{i \neq j} X_i X_j) - n^2 p^2 \\
            &= np + n(n-1)p^2 - n^2 p^2 = np(1-p)
        \end{aligned}
        \end{equation}
    \end{solution}
\end{example}

\section{Independence}
\begin{definition}[Definition of Independence]
    Two RVs $X$ and $Y$ are independent if
    \begin{align}
        p_{X, Y}(x, y) = p_{X}(x) \cdot p_{Y}(y)
    \end{align}
    for all $x$ and $y$.
\end{definition}
\begin{property}[Properties of Independence] ~
    \begin{itemize}
        \item $\mathbf{E}[g(X)h(Y)] = \mathbf{E}[g(X)] \cdot \mathbf{E}[h(Y)]$. (go back to the definition of expectation, divide the sum into two parts, and you will see the independence.)
        \item $\text{var}(g(X) + h(Y)) = \text{var}(g(X)) + \text{var}(h(Y))$. (use $\text{var}(A + B) = \mathbf{E}[(A + B)^2] - \mathbf{E}[A + B]^2$.)
    \end{itemize}
\end{property}

\begin{definition}[Definition of Conditional Independence]
    Two RVs $X$ and $Y$ are conditionally independent given $A$ if
    \begin{align}
        p_{X, Y|A}(x, y) = p_{X|A}(x) p_{Y|A}(y)
    \end{align}
    for all $x$ and $y$.
\end{definition}
\begin{example}[Data packet problem]
    A network system with n nodes randomly redistributes each node's data packet through a central processor. Let $X$ be the number of nodes that receive their original data packets. Find $\mathbf{E}[X]$ and $\text{var}(X)$. 
    \begin{solution}
        Define $X_i$ as the indicator of the $i$-th node. Then $X = \sum_{i=1}^n X_i$. 
        \begin{equation}
            \mathbf{E}[X] = \sum_{i=1}^{n} \mathbf{E}[X_i] = n \cdot \frac{1}{n} = 1
        \end{equation}
        \begin{equation}
        \begin{aligned}
            \text{var}(X) &= \mathbf{E}[X^2] - \mathbf{E}[X]^2 = \sum_{i=1}^{n} \mathbf{E}[X_i^2] + \sum_{i \neq j} \mathbf{E}[X_i X_j] - 1 \\ 
            &= n \cdot \frac{1}{n} + \sum_{i=1}^{n} \left(\mathbf{P}(X_i = 1) \cdot \mathbf{P}(X_j = 1 | X_i = 1)\right) - 1 \\ 
            &= n(n-1) \cdot \frac{1}{n(n-1)} = 1
        \end{aligned}
        \end{equation}
    \end{solution}
\end{example}

\chapter{Continuous Random Variables}


\section{Continuous RVs and PDFs}

\begin{definition}[Continuous Random Variables]
    A RV $X$ is continuous if there exists a nonnegative function $f_{X}(x)$, called \textbf{probability density function}, such that
    \begin{align}
        \mathbf{P}(X \in A) = \int_{A} f_{X}(x) \dd{x}
    \end{align}
    Especially, if $A \subseteq \mathcal{B}(\mathbb{R})$, then
    \begin{align}
        \mathbf{P}(a \leq X \leq b) = \int_{a}^{b} f_{X}(x) \dd{x}
    \end{align}
\end{definition}

\begin{property}[Properties of PDF]
    Let $X$ be a continuous RV with PDF $f_{X}(x)$, then
    \begin{itemize}
        \item $f_{X}(x) \geq 0$, and $f_{X}(x)$ can be any nonnegative number. (even $+\infty$)
        \item $\int_{-\infty}^{\infty} f_{X}(x) \dd{x} = 1$.
        \item $\mathbf{P}(x \leq X \leq x + \delta) = \int_{x}^{x + \delta} f_{X}(x) \dd{x} \approx f_{X}(x) \cdot \delta$.
    \end{itemize}
\end{property}


\section{Expectation and Variance}

\begin{definition}[Definitions of Expectation and Variance]
    Let $X$ be a continuous RV with PDF $f_{X}(x)$, then
    \begin{equation}
    \begin{aligned}
        &\mathbf{E}[X] = \int_{-\infty}^{\infty} x f_{X}(x) \dd{x} \\ 
        &\mathbf{E}[g(X)] = \int_{-\infty}^{\infty} g(x) f_{X}(x) \dd{x}
    \end{aligned}
    \end{equation}
    \begin{equation}
    \begin{aligned}
        \text{var}(X) &= \mathbf{E}[(X - \mathbf{E}[X])^{2}] = \int_{-\infty}^{\infty} (x - \mathbf{E}[X])^{2} f_{X}(x) \dd{x} \\ 
        &= \mathbf{E}[X^{2}] - \mathbf{E}[X]^{2}
    \end{aligned}
    \end{equation}
\end{definition}

\begin{example}[Uniform RV]
    Consider a RV $X$ that takes value in an interval $[a, b]$ with PDF
    \begin{figure}[H]
        \centering
        \includegraphics[width=0.3\textwidth]{images/Note-6.1.png}
    \end{figure}
    \vspace{-2em}
    \begin{equation}
        \mathbf{E}[X] = \int_{a}^{b} \frac{x}{b - a} \dd{x} = \frac{a + b}{2}
    \end{equation}
    \begin{equation}
        \text{var}(X) = \mathbf{E}[X^2] - (\mathbf{E}[X])^2 = \int_{a}^{b} \frac{x^2}{b - a} \dd{x} - \left(\frac{a + b}{2}\right)^2 = \frac{(b - a)^2}{12}
    \end{equation}
\end{example}

\begin{example}[Exponential RV]
    Consider an exponential RV $X$ that has a PDF of the form
    \begin{equation}
        f_{X}(x) = \lambda \e^{-\lambda x},~ x \geq 0
    \end{equation}
    \begin{equation}
        \mathbf{E}[X] = \int_{0}^{\infty} x \lambda \e^{-\lambda x} \dd{x} = (-x \e^{-\lambda x}) \Big|_{0}^{\infty} + \int_{0}^{\infty} \e^{-\lambda x} \dd{x} = \frac{1}{\lambda}
    \end{equation}
    \begin{equation}
        \text{var}(X) = \mathbf{E}[X^2] - (\mathbf{E}[X])^2 = (-x^2 \e^{-\lambda x}) \Big|_{0}^{\infty} + \int_{0}^{\infty} 2x \e^{-\lambda x} \dd{x} - \frac{1}{\lambda^2} = \frac{1}{\lambda^2}
    \end{equation}
\end{example}


\section{Comulative Distribution Function}
We want to describe both discrete and continuous RVs in a unified way. An intuition is to ``accumulate'' the probability ``up to'' the value $x$.

\begin{definition}[CDF]
    The cumulative distribution function (CDF) of a RV $X$ is defined as
    \begin{align}
        F_{X}(x) = \mathbf{P}(X \leq x) = \left\{
        \begin{aligned}
            &\sum_{t \leq x} p_{X}(t), &\text{discrete} \\ 
            &\int_{-\infty}^{x} f_{X}(t) \dd{t}, &\text{continuous}
        \end{aligned}
        \right.
    \end{align}
\end{definition}

\begin{example}[Some examples of CDF] ~
    \begin{figure}[H]
        \centering
        \includegraphics[width=0.6\textwidth]{images/Note-6.2.png}
        \caption{CDF: Continuous RV}
    \end{figure}
    \begin{figure}[H]
        \centering
        \includegraphics[width=0.6\textwidth]{images/Note-6.3.png}
        \caption{CDF: Discrete RV}
    \end{figure}
\end{example}

\begin{property}[Properties of CDF] ~
    \begin{itemize}
        \item Monotonically nondecreasing: $x \leq y \Rightarrow F_{X}(x) \leq F_{X}(y)$. (since $p_{X}(x) \geq 0$ and $f_{X}(x) \geq 0$)
        \item $F_X(x) \to 0$ as $x \to -\infty$ and $F_X(x) \to 1$ as $x \to \infty$.
        \item When $X$ is continuous: $F_{X}(x)$ is a continuous function, $f_{X}(x) = \dv{F_X}{x}$.
        \item When $X$ is discrete: $F_{X}(x)$ i piecewise constant function, $p_{X}(k) = F_{X}(k) - F_{X}(k-1)$.
    \end{itemize}
\end{property}

\begin{example}[CDF of Geometry RV]
    Let $X$ be a geometry RV with parameter $p$, then
    \begin{equation}
    \begin{aligned}
        &\mathbf{P}(X = k) = p(1 - p)^{k - 1} \\ 
        \Rightarrow &F_{X}(x) = \sum_{k \leq x} p(1 - p)^{k - 1} = 1 - (1 - p)^{\lfloor x \rfloor}
    \end{aligned}
    \end{equation}
\end{example}

\begin{example}[Maximum test score]
    Assume that each test takes one of the values from 1 to 10 with equal probability 1/10 independently. What is the PMF of the maximum of three test scores?
    \begin{equation}
        X = \max\{X_1, X_2, X_3\}
    \end{equation}
    \begin{solution}
        We have the CDF of $X$ as
        \begin{equation}
        \begin{aligned}
            F_{X}(x) &= \mathbf{P}(X \leq x) \\ 
            &= \mathbf{P}(\max\{X_1, X_2, X_3\} \leq x) \\ 
            &= \mathbf{P}(X_1 \leq x, X_2 \leq x, X_3 \leq x) \\ 
            &= \mathbf{P}(X_1 \leq x) \cdot \mathbf{P}(X_2 \leq x) \cdot \mathbf{P}(X_3 \leq x) \\
            &= F_{X_1}(x) \cdot F_{X_2}(x) \cdot F_{X_3}(x) \\ 
            &= \left(\frac{\lfloor x \rfloor}{10}\right)^3
        \end{aligned}
        \end{equation}
        and the PMF of $X$ is
        \begin{equation}
            p_{X}(x) = F_{X}(x) - F_{X}(x - 1) = \left(\frac{\lfloor x \rfloor}{10}\right)^3 - \left(\frac{\lfloor x \rfloor - 1}{10}\right)^3
        \end{equation}
        Similarly, this approach can be introduced to the minimum of three test scores (need to reverse by $1 - \mathbf{P}(xxx)$).
    \end{solution}
\end{example}


\section{Normal and Exponential RVs}

\begin{definition}[Gaussian (Normal) RV] ~
    \begin{itemize}
        \item Standard normal $N(0, 1)$:
        \begin{equation}
            f_{X}(x) = \frac{1}{\sqrt{2\pi}} \e^{-x^2/2}
        \end{equation}
        which has $\mathbf{E}[X] = 0$ and $\text{var}(X) = 1$. (intergrate by part)
        \item General normal $N(\mu, \sigma^2)$:
        \begin{equation}
            f_{X}(x) = \frac{1}{\sqrt{2\pi} \sigma} \e^{-(x - \mu)^2/2\sigma^2}
        \end{equation}
        which has $\mathbf{E}[X] = \mu$ and $\text{var}(X) = \sigma^2$. (let $Y = (X - \mu) / \sigma$ and $Y$ is standard normal)
        \item The CDF of standard normal is hard to calculate, we denote it as 
        \begin{equation}
            \Phi(x) = \frac{1}{\sqrt{2\pi}} \int_{-\infty}^{x} \e^{-t^2/2} \dd{t}    
        \end{equation}
        If $X \sim N(\mu, \sigma^2)$, then $\mathbf{P}(X \leq x) = \mathbf{P}(\frac{X - \mu}{\sigma} \leq \frac{x - \mu}{\sigma}) = \Phi(\frac{x - \mu}{\sigma})$.
    \end{itemize}
\end{definition}

\begin{definition}[Exponential RV]
    A RV $X$ is exponential with parameter $\lambda$ if it has PDF
    \begin{equation}
        f_{X}(x) = \lambda \e^{-\lambda x},~ x \geq 0
    \end{equation}
    which has $\mathbf{E}[X] = 1 / \lambda$ and $\text{var}(X) = 1 / \lambda^2$, and the CDF is
    \begin{equation}
        F_{X}(x) = 1 - \int_{x}^{+\infty} \lambda \e^{-\lambda t} \dd{t} = 1 - \e^{-\lambda x}
    \end{equation}
\end{definition}

A continuous exponential RV also has the memoryless property as discrete exponential RV, which means
\begin{property}[Memoryless Property]
    Let $X$ be an exponential RV with parameter $\lambda$, then for all $x, c \geq 0$,
    \begin{equation}
        \mathbf{P}(X > x + c \mid X > c) = \mathbf{P}(X > x) = \e^{-\lambda x}
    \end{equation}
\end{property}

Because $\e^{-\lambda x} \sim x + 1, ~x \to 0$, we have
\begin{property}[Continuous-time analog of the geometric] ~
    \begin{itemize} 
        \item $\mathbf{P}(0 \leq X \leq \delta) = 1 - \e^{-\lambda \delta} \approx \lambda \delta$.
        \item $\mathbf{P}(c \leq X \leq c + \delta \mid X > c) \approx \lambda \delta$.
    \end{itemize}
\end{property}

\input{chaps/chap_Multiple_Continuous_Random_Variables.tex}
\input{chaps/chap_Derived_Distributions_and_Entropy.tex}


\chapter{Convolution, Covariance, Correlation, and Conditional Expectation}

\section{Convolution}
\begin{definition}[Convolution]
    The convolution of two functions $f$ and $g$ is defined as
    \begin{equation}
        (f * g)(x) = \int_{-\infty}^{+\infty} f(x') g(x - x') \dd{x'}
    \end{equation}
    The convolution of two discrete functions is defined as
    \begin{equation}
        (f * g)(k) = \sum_{k' = -\infty}^{+\infty} f(k') g(k - k')
    \end{equation}
\end{definition}
When considering a RV $W = X + Y$ where $X$ and $Y$ are independent discrete RVs
\begin{equation}
    p_{W}(w) = \mathbf{P}(X + Y = w) = \sum_{\{(x, y): x + y = w\}} p_{X}(x) p_{Y}(y) = \sum_{x} p_{X}(x) p_{Y}(w - x)
\end{equation}
For continuous cases
\begin{equation}
    f_{W}(w) = \mathbf{P}(X + Y = w) = \int_{-\infty}^{+\infty} f_{X}(x) f_{Y}(w - x) \dd{x}
\end{equation}
If $X$ and $Y$ are dependent, let $f_{X, Y}(x, y)$ be the joint PDF of $X$ and $Y$, then
\begin{equation}
    f_{W}(w) = \int_{-\infty}^{+\infty} f_{X}(x) f_{X, Y}(x, w - x) \dd{x}
\end{equation}
\begin{remark}
    This is somehow similar to LTI systems in SS. When the system is not time-invariant, $h(t - \tau)$ becomes $h(t, \tau)$.
\end{remark}
\begin{remark}
    \begin{itemize}
    \item Conditional CDF: $F_{W \mid X}(w \mid x) = F_{Y}(w - x)$
    \item Conditional PDF: $f_{W \mid X}(w \mid x) = f_{Y}(w - x)$
    \item Joint PDF: $f_{W, X}(w, x) = f_{X}(x) f_{Y}(w - x)$
    \item Marginal PDF: $f_{W}(w) = \int_{-\infty}^{+\infty} f_{W, X}(w, x) \dd{x} = \int_{-\infty}^{+\infty} f_{X}(x) f_{Y}(w - x) \dd{x}$
    \end{itemize}
\end{remark}
\begin{example}[Convolution of Uniform RVs]
    The RVs $X$ and $Y$ are independent and uniformly distributed in the interval $[0, 1]$. Find the PDF of $Z = X +Y$.
\end{example}
\begin{solution}
    \begin{equation}
        f_{Z}(z) = \int_{-\infty}^{+\infty} f_{X}(x) f_{Y}(z - x) \dd{x} = \left\{\begin{aligned}
            &0, \quad &z < 0 \\ 
            &z, \quad &0 \leq z < 1 \\ 
            &2 - z, \quad &1 \leq z < 2 \\
            &0, \quad &z \geq 2
        \end{aligned}\right.
    \end{equation}
\end{solution}

\begin{example}[Sum of Two Independent Normal RVs]
    $X \sim N(\mu_x, \sigma_x^2)$, $Y \sim N(\mu_y, \sigma_y^2)$ are independent. 
    \begin{itemize}
        \item The joint PDF is
        \begin{equation}
            f_{X, Y}(x, y) = \frac{1}{2 \pi \sigma_x \sigma_y} \e^{-\frac{(x - \mu_x)^2}{2 \sigma_x^2}} \e^{-\frac{(y - \mu_y)^2}{2 \sigma_y^2}}
        \end{equation}
        which is a constant (as shown in Figure \ref{fig:equal-probability-ellipse}) on the ellipse where
        \begin{equation}
            \frac{(x - \mu_x)^2}{2\sigma_x^2} + \frac{(y - \mu_y)^2}{2\sigma_y^2} = const.
        \end{equation}
        \item Let $W = X + Y$, the PDF of $W$ is
        \begin{equation}
            f_{W}(w) = \int_{-\infty}^{+\infty} f_{X, Y}(x, w - x) \dd{x} = \frac{1}{2\pi\sigma_x\sigma_y}\int_{-\infty}^{+\infty} \exp\left(-\frac{(x - \mu_x)^2}{2\sigma_x^2} -\frac{(w - x - \mu_y)^2}{2\sigma_y^2}\right) \dd{x}
            = \cdots
        \end{equation}
        which is also a normal RV with mean $\mu_w = \mu_x + \mu_y$ and variance $\sigma_w^2 = \sigma_x^2 + \sigma_y^2$.
    \end{itemize}
    \begin{proof}
        Obvious after Fourier transformation.
    \end{proof}
    \begin{figure}[H]
        \centering
        \includegraphics[width=0.4\textwidth]{images/Note-9.1.png}
        \caption{The joint PDF of two independent normal RVs.}
        \label{fig:equal-probability-ellipse}
    \end{figure}
\end{example}
\begin{example}[The Difference of Two Independent Exponential RVs]
    Consider the case where $X$ and $Y$ are independent exponential RVs with parameter $\lambda$ and $\mu$, respectively. Find the PDF of $Z = X - Y$.
\end{example}
\begin{solution}
    \begin{equation}
        f_{Z}(z) = f_{X-Y}(z) = \int_{-\infty}^{+\infty} f_{X}(x) f_{-Y}(z - x) \dd{x} = \int_{-\infty}^{+\infty} f_{X}(x) f_{Y}(x - z) \dd{x}
    \end{equation}
    For $z \geq 0$, $f_{X}(x)$ is nonzero when $x \geq 0$ and $f_{Y}(x - z)$ is nonzero when $x \geq z$.
    \begin{equation}
        f_{Z}(z) = \int_{z}^{+\infty} \lambda \e^{-\lambda x} \mu \e^{-\mu (x - z)} \dd{x} = \int_{z}^{+\infty} \lambda \mu \e^{-(\lambda + \mu)x + \mu z} \dd{x} = \frac{\lambda \mu}{\lambda + \mu} \e^{-\lambda z} \quad (z \geq 0)
    \end{equation}
    For $z < 0$, $f_{X}(x)$ is nonzero when $x \geq 0$ and $f_{Y}(x - z)$ is nonzero when $x \leq z$.
    \begin{equation}
        f_{Z}(z) = \int_{0}^{z} \lambda \e^{-\lambda x} \mu \e^{-\mu (x - z)} \dd{x} = \int_{0}^{z} \lambda \mu \e^{-(\lambda + \mu)x + \mu z} \dd{x} = \frac{\lambda \mu}{\lambda + \mu} \e^{\mu z} \quad (z < 0)
    \end{equation}
\end{solution}


\section{Covariance and Correlation}
\begin{definition}[Covariance]
    The covariance of two RVs $X$ and $Y$ is defined as
    \begin{equation}
        \cov(X, Y) = \E[(X - \mu_X)(Y - \mu_Y)] = \E[XY] - \mu_X \mu_Y
    \end{equation}
    where $\mu_X = \E[X]$ and $\mu_Y = \E[Y]$. \\ 
    The covariance describes the degree to which two RVs vary together. If $X$ and $Y$ are independent, then $\cov(X, Y) = 0$. 
\end{definition}
\begin{proof}
    The second equality is because
    \begin{equation}
        \E[(X - \mu_X)(Y - \mu_Y)] = \E[XY] - \mu_X \E[Y] - \mu_Y \E[X] + \mu_X \mu_Y
        = \E[XY] - \mu_X \mu_Y
    \end{equation}
\end{proof}
\begin{property}[Properties of Covariance]
    \begin{itemize}
        \item $\cov(X, X) = \var(X)$
        \item $\cov(X, aY + b) = a \cov(X, Y)$ (linearity of expectation)
        \item $\cov(X, Y + Z) = \cov(X, Y) + \cov(X, Z)$ (linearity of expectation)
        \item independent $\Rightarrow \cov(X, Y) = 0$ (but not the other way around)
    \end{itemize}
\end{property}

\begin{theorem}[Variance of the Sum of Multiple RVs]
    \begin{equation}
        \var\qty(\sum_{i=1}^{n} X_i) = \sum_{i=1}^{n} \var(X_i) + \sum_{\{(i,j): j \neq i\}}^{n} \cov(X_i, X_j)
    \end{equation}
    which is somehow similar to the complete square of the quadratic form.
\end{theorem}
\begin{proof}
    Denote $\widetilde{X_i} = X_i - \E[X_i]$, then
    \begin{equation}
        \E\qty[\widetilde{X_i}^2] = \E\qty[X_i^2 - 2X_i\E[X_i] + \E[X_i]^2] = \E[X_i^2] - 2\E[X_i]\E[X_i] + \E[X_i]^2 = \var(X_i)
    \end{equation}
    \begin{equation}
        \E\qty[\widetilde{X_i}\widetilde{X_j}] = \E\qty[X_i X_j - X_i \E[X_j] - \E[X_i] X_j + \E[X_i] \E[X_j]] = \E[X_i X_j] - \E[X_i]\E[X_j] = \cov(X_i, X_j)
    \end{equation} 
    Therefore
    \begin{equation}
    \begin{aligned}
        \var\qty(\sum_{i=1}^{n} X_i) &= \E\qty[\qty(\sum_{i=1}^{n}X_i - \E[X_i])^2] = \E\qty[\qty(\sum_{i=1}^{n} \widetilde{X_i})^2] \\ 
        &= \sum_{i=1}^{n}\E\qty[\widetilde{X_i}] + \sum_{\{(i,j): j \neq i\}}^{n}\E\qty[X_i, X_j] \\ 
        &= \sum_{i=1}^{n} \var(X_i) + \sum_{\{(i,j): j \neq i\}}^{n} \cov(X_i, X_j)
    \end{aligned}
    \end{equation}
\end{proof}

\begin{example}[Data Packet Problem]
    Consider the data packet problem again, where a router randomly distributes $n$ data packets to $n$ nodes. Find the variance of $X$, the number of nodes where the $i$-th node receives the $i$-th data packet.
\end{example}
\begin{solution}
    Let $X_i$ be the indicator of the event that the $i$-th node receives the $i$-th data packet. Then $X = \sum_{i=1}^{n} X_i$. Note that $X_i$ obeys the Bernoulli distribution with parameter $p = 1/n$. Therefore, 
    \begin{equation}
        \E[X_i] = \frac{1}{n}, \quad \var(X_i) = \frac{1}{n}\left(1 - \frac{1}{n}\right)
    \end{equation}
    For $i \neq j$
    \begin{equation}
    \begin{aligned}
        \cov(X_i, X_j) &= \E[X_i X_j] - \E[X_i]\E[X_j] = \mathbf{P}(X_i = 1, X_j = 1) - \frac{1}{n^2} \\ 
        &= \frac{1}{n(n-1)} - \frac{1}{n^2} = \frac{1}{n^2(n-1)}
    \end{aligned}
    \end{equation}
    Therefore
    \begin{equation}
    \begin{aligned}
        \var(X) &= \sum_{i=1}^{n} \var(X_i) + \sum_{\{(i,j): j \neq i\}}^{n} \cov(X_i, X_j) \\ 
        &= n \cdot \frac{1}{n}\left(1 - \frac{1}{n}\right) + n(n-1) \cdot \frac{1}{n^2(n-1)} = 1
    \end{aligned}
    \end{equation}
\end{solution}

\begin{definition}[Coefficient of Correlation]
    The coefficient of correlation of two RVs $X$ and $Y$ is defined as
    \begin{equation}
        \rho(X, Y) = \frac{\cov(X, Y)}{\sigma_X\sigma_Y} = \E\left[\frac{X - \mu_X}{\sigma_X} \cdot \frac{Y - \mu_Y}{\sigma_Y}\right]
    \end{equation}
    where $\sigma_X = \sqrt{\var(X)}$, $\sigma_Y = \sqrt{\var(Y)}$, $\mu_X = \E[X]$, and $\mu_Y = \E[Y]$. 
\end{definition}
According to the linearity of covariance, if $X \to 1000X$ (e.g. change the unit of $X$ from g to kg), then $\cov(X, Y) \to 1000 \cov(X, Y)$, which is not what we want. So we need to normalize the covariance. Here we choose $\sigma_X$ and $\sigma_Y$ as the normalization factors because $\sigma_X \to a\sigma_X$ when $X \to aX + b$, i.e. $\sigma$ is proportional to the scale and unrelated to the shift, just like the covariance.
\begin{property}[Properties of Coefficient]
    \begin{itemize}
        \item $\abs{\rho(X, Y)} \leq 1$
        \item $\abs{\rho(X, Y)} = 1 \Leftrightarrow (X - \E[X]) = c(Y - \E[Y])$ (linearly related)
        \item $X$ and $Y$ are independent $\Rightarrow \rho(X, Y) = 0$ (but not the other way around)
    \end{itemize}
\end{property}

\begin{example}[Data Transmission]
    Consider $n$ independent transmissions of a binary signal where the probability of transmitting a bit 1 is $p$ and a bit 0 is $1 - p$. Let $X$ represent the number of 1's transmitted and $Y$ the number of 0's. Find the correlation coefficient between $X$ and $Y$ .
\end{example}
\begin{solution}
    Since $X + Y = n$ which means $X$ and $Y$ are linearly related, we have $\rho(X, Y) = -1$. \\ 
    Or, we have $X + Y = n$ and $\E[X] + \E[Y] = n$, thus
    \begin{equation}
    \begin{aligned}
        &X - \E[X] = -(Y - \E[Y]) \\ 
        \Rightarrow \quad &\cov(X, Y) = \E[(X - \E[X])(Y - \E[Y])] = -\var(X) = -\var(Y) \\ 
        \Rightarrow \quad &\rho(X, Y) = \frac{\cov(X, Y)}{\sigma_X \sigma_Y} = -\frac{\var(X)}{\sqrt{\var(X)}\sqrt{\var(Y)}} = -1
    \end{aligned}
    \end{equation}
\end{solution}


\section{Conditional Expectation}
\begin{theorem}[The Conditional Expectation as an Estimator]
    Denote the conditional expectation
    \begin{equation}
        \hat{X} = \E[X \mid Y]
    \end{equation}
    as an estimator of $X$ given $Y$, and the estimation error as 
    \begin{equation}
        \widetilde{X} = X - \hat{X}
    \end{equation}
    is a RV.
\end{theorem}
\begin{property}[Properties of the Estimator]
    \begin{itemize}
        \item Unbiased: $\E[\widetilde{X}] = \E[\E[\widetilde{X} \mid Y]] = \E[\E[X \mid Y] - \E[\hat{X} \mid Y]] = \E[\hat{X} - \hat{X}] = 0$
        \item Uncorrelated: $\cov(\hat{X}, \widetilde{X}) = \E[\hat{X}\widetilde{X}] - \E[\hat{X}]\E[\widetilde{X}] = \E[\E[\hat{X}\widetilde{X} \mid Y]] = \E[\hat{X}\E[\widetilde{X} \mid Y]] = \E[\hat{X} \cdot 0] = 0$
        \item $\var(X) = \var(\hat{X}) + \var(\widetilde{X}) + 2\cov(\hat{X}, \widetilde{X}) = \var(\hat{X}) + \var(\widetilde{X})$
    \end{itemize}
    The second to last equality of the second property is because $\hat{X}$ is fully determined by $Y$, i.e. given $Y$, $\hat{X}$ is a number.
\end{property}

The first item of the third property of the estimator $\var(\hat{X})$ can be written as 
\begin{equation}
    \var(\hat{X}) = \var\qty(\E[X \mid Y])
\end{equation}
The second item of the third property of the estimator $\var(\widetilde{X})$ can be written as
\begin{equation}
    \var(\widetilde{X}) = \E\qty[\widetilde{X}^2] - \E\qty[\widetilde{X}]^2 = \E\qty[\E\qty[\widetilde{X}^2 \mid Y]] = \E\qty[\E\qty[\qty(X - \hat{X})^2 \mid Y]] = \E[\var(X \mid Y)]
\end{equation}
So we get the total variance decomposition
\begin{theorem}[Law of Total Variance]
    \begin{equation}
        \var(X) = \var\qty(\E[X \mid Y]) + \E\qty[\var(X \mid Y)]
    \end{equation}
\end{theorem}


\appendix
\chapter{Problem Solutions}


\section{Derived Distributions}
\textbf{In the lecture right after I finished this section, the lecturer provided an ``official'' solution to this problem, see section \ref{sec:derived-distributions}.}

In the homework, I constantly encounter some problems like this:
\begin{example}[PDF of a Function of Multiple Continuous RVs]
    Let $X, Y$ be two continuous RVs with joint PDF $f_{X, Y}(x, y)$, and $Z = g(X, Y)$. Find the PDF of $Z$.
\end{example}

Here I introduce two approaches to solve this kind of problems: CDF and Jacobian transformation.
\begin{solution}
    \textbf{CDF:}
    \begin{enumerate}
        \item Find the CDF of $Z$
            \begin{equation}
                F_{Z}(z) = \mathbf{P}(Z \leq z) = \mathbf{P}(g(X, Y) \leq z) = \iint_{g(x, y) \leq z} f_{X, Y}(x, y) \dd{x} \dd{y}
            \end{equation}
        \item Differentiate $F_{Z}(z)$ to get the PDF of $Z$
            \begin{equation}
                f_{Z}(z) = \dv{F_{Z}(z)}{z}
            \end{equation}
    \end{enumerate}
\end{solution}
\begin{solution}
    \textbf{Jacobian Transformation:} This approach requires finding a valid variable substitution with invertibility. \\ 
    Take $Z = g(X, Y) = X + Y$ as an example. Normally, if we want to calculate the determinant of the Jacobian matrix, the Jacobian matrix must be square. However, here $Z$ is a scalar instead of a vector. We take the following steps to avoid this problem:
    \begin{enumerate}
        \item Let $U = X + Y$ and $V = X$, then $X = V$ and $Y = U - V$.
        \item Calculate the Jacobian matrix and its determinant
            \begin{align}
                &J = \mqty(\pdv{X}{U} & \pdv{X}{V} \\ \pdv{Y}{U} & \pdv{Y}{V}) = \mqty(0 & 1 \\ 1 & -1) \\ 
                &\det(J) = -1 
            \end{align}
        \item Find the joint PDF of $U$ and $V$
            \begin{equation}
                f_{U, V}(u, v) = f_{X, Y}(v, u - v) \cdot \abs{\det(J)} = f_{X, Y}(v, u - v)
            \end{equation}
        \item Integrate out $V$ to get the marginal PDF of $U$, which is the PDF of $Z$
            \begin{equation}
                f_{U}(u) = \int_{-\infty}^{+\infty} f_{X, Y}(v, u - v) \dd{v}
            \end{equation}
    \end{enumerate}
\end{solution}



\end{document}
