\documentclass[device=normal, lang=en]{elegantbook}
\usepackage{amsmath}
\usepackage{amssymb}
\usepackage{float}
\usepackage{extarrows}
\usepackage{physics}

\definecolor{pgcolor}{RGB}{251, 250, 248}
\pagecolor{pgcolor}
\numberwithin{equation}{section}

% \theoremstyle{definition} %
% \newtheorem{property}{Property}[section] %

\title{Probability Notes}
\author{FHYQ-Dong}
\date{\today}
\version{8.0}
\cover{images/Note-cover.png}
\definecolor{customcoverlinecolor}{RGB}{82, 59, 148}
\colorlet{coverlinecolor}{customcoverlinecolor}


\begin{document}

\maketitle
\frontmatter

\tableofcontents
\mainmatter


\chapter{Probability Space}
\begin{definition}[Probability Space]
    Probability space is a tuple $\left(\varOmega,~\mathcal{F},~\mathbf{P}\right)$ where
    \begin{itemize}
        \item Sample space $\varOmega$ is a set of all possible outcomes.
        \item $ \sigma $-algebra $\mathcal{F}$ is a collection of subsets of $\varOmega$.
        \item Probability measure $\mathbf{P}$ is a function that assigns a probability to each event in $\mathcal{F}$.
    \end{itemize}
\end{definition}

\begin{definition}[Sample Space]
    Sample space $\varOmega$ is a set of all possible outcomes, which should satisfy the following properties:
    \begin{itemize}
        \item Mutually Exclusive: The elements in $\varOmega$ should be unique.
        \item Collectively Exhaustive: The elements in $\varOmega$ should cover all possible outcomes.
    \end{itemize}
\end{definition}

Once the experiment is conducted, there is \textbf{exactly one} element in the sample space $\varOmega$ occurs.

\begin{example}[Example of Sample Space]
    Here are some examples of sample space:
    \begin{itemize}
        \item Discrete: $\varOmega = \{1, 2, 3, 4, 5, 6\}$ - Rolling a fair die.
        \item Continuous: $\varOmega = [0, 1]$ - Randomly selecting a real number between 0 and 1.
    \end{itemize}
\end{example}

Before we define the $\sigma$-algebra, we need to introduce the concept of \textbf{countable}. We say a set $X$ is countable if there is a bijection between $X$ and $\mathbb{N}$. Some countable sets are: $\mathbb{N}$, $\mathbb{Z}$, $\mathbb{Q}$, $\mathbb{N}^2$, $\mathbb{Z}^2$, $\mathbb{Q}^2$, $\cup_{n=1}X_{n}$. Some uncountable sets are: $\mathbb{R}$, $\mathbb{C}$, $\mathbb{R}^2$, $\mathbb{C}^2$.

\begin{definition}[$\sigma$-algebra]
    We say a set $\mathcal{F}$  collection of subsets of $\varOmega$ is a $\sigma$-algebra if it satisfies the following properties:
\end{definition}

\begin{remark}
    Not all the things in real worlf can be measured by probability, especially when things are continuous (uncountable).
\end{remark}

\chapter{Conditional Probability}
\section{Conditional Probability}
\begin{definition}[Conditional Probability]
    Conditional probability is the probability of an event $A$ given that another event $B$ has already occurred. It is denoted by $P(A|B)$ and is defined as:
    \begin{equation}
        \mathbf{P}(A|B) = \frac{\mathbf{P}(A \cap B)}{\mathbf{P}(B)}
    \end{equation}
    Note that when $\mathbf{P}(B) = 0$, the conditional probability is undefined.
\end{definition}

We can easily check the conditional probability satisfies the properties of probability measure, which means it is a legitimate probability on a new universe.
\begin{itemize}
    \item Non-negativity: $\mathbf{P}(A|B) \geq 0$.
    \item Normalization: $\mathbf{P}(\varOmega|B) = 1$.
    \item Additivity: when $A$ and $B$ are disjoint, $\mathbf{P}(A \cup B|C) = \frac{\mathbf{P}((A \cup B) \cap C)}{\mathbf{P}(C)} = \frac{\mathbf{P}(A \cap C) + \mathbf{P}(B \cap C)}{\mathbf{P}(C)} = \mathbf{P}(A|C) + \mathbf{P}(B|C)$. (The second equality is due to $A \cap C$ and $B \cap C$ are disjoint).
\end{itemize}

\begin{example}[Discrete and Continuous]
    For discrete case, the conditional probability can be calulated by:
    \begin{equation}
        \mathbf{P}(A|B) = \frac{\#~of~elements~in~A}{\#~of~elements~in~B}
    \end{equation}
    For continuous case, the conditional probability can be calculated by:
    \begin{equation}
        \mathbf{P}(A|B) = \frac{area~of~A}{area~of~B}
    \end{equation}
\end{example}

When solving a problem, the following equations may help:
\begin{align}
    \mathbf{P}(A \cap D) &= \mathbf{P}(A) \cdot \mathbf{P}(D|A) \\
    &= \mathbf{P}(D) \cdot \mathbf{P}(A|D) \\
    \mathbf{P}(A \cap B \cap C) &= \mathbf{P}(A) \cdot \mathbf{P}(B|A) \cdot \mathbf{P}(C|A \cap B)
\end{align}
For the second equation: 1.event $A$ occurs; 2.event $B$ occurs given $A$; 3.event $C$ occurs given $A$ and $B$. Or
\begin{align*}
    \mathbf{P}(A \cap B \cap C) &= \mathbf{P}(A \cap B) \cdot \mathbf{P}(C | A \cap B) \\ 
    &= \mathbf{P}(A) \cdot \mathbf{P}(B | A) \cdot \mathbf{P}(C | A \cap B)
\end{align*}
\begin{remark}
    These equations are not fixed, depending on $\mathbf{P}(A)$, $\mathbf{P(A \cap B)}$ and $\mathbf{P}(A | B)$ which are easier to calculate.
\end{remark}

\section{Total Probability Theorem}
We can calculate the total probability with \textit{Devide and Conquer} strategy:
\begin{enumerate}
    \item Devide event $B$ by $A_1, A_2, \cdots A_n$, note that $\bigcup_{i=1}^n A_i = \varOmega$ and $A_i \cap A_j = \emptyset$.
    \item Calculate $\mathbf{P}(B | A_i)$ which are easier to calculate.
    \item Sum all the probabilities.
\end{enumerate}
\begin{figure}[H]
    \centering
    \includegraphics[width=0.3\textwidth]{images/Note-2.1.png}
    \caption{Total Probability Theorem}
    \label{fig:total-probability}
\end{figure}

\begin{theorem}[Total Probability Theorem]
    Let $A_1, A_2, \cdots, A_n$ be a partition of the sample space $\varOmega$. Then for any event $B$,
    \begin{equation}
        \mathbf{P}(B) = \sum_{i=1}^{n} \mathbf{P}(A_i) \cdot \mathbf{P}(B | A_i)
    \end{equation}
\end{theorem}
\begin{example}[Die Rolling]
    You roll a fair four-sided die. If the result is 1 or 2, you roll once more but otherwise, you stop. What is the probability that the sum total of rolls is at least 4?
    \begin{solution}
        \begin{enumerate}
            \item Let $A_i = \{the~first~roll~is~i\}, B = \{the~sum~total~of~rolls~is~at~least~4\}$.
            \item $\mathbf{P}(A_1) = \mathbf{P}(A_2) = \mathbf{P}(A_3) = \mathbf{P}(A_4) = \frac{1}{4}$.
            \item $\mathbf{P}(B | A_1) = \frac{1}{2}, \mathbf{P}(B | A_2) = \frac{3}{4}, \mathbf{P}(B | A_3) = 0, \mathbf{P}(B | A_4) = 1$.
            \item $\mathbf{P}(B) = \frac{9}{16}$.
        \end{enumerate}
    \end{solution}
\end{example}


\section{Bayes' Theorem}
If we know:
\begin{itemize}
    \item ``Prior'' probabilities $\mathbf{P}(A_i)$.
    \item ``Likelihood'' probabilities $\mathbf{P}(B | A_i)$.
\end{itemize}
We wish we can calculate the ``Posterior'' probabilities $\mathbf{P}(A_i | B)$. 

\begin{theorem}[Bayes' Theorem]
    Let $A_1, A_2, \cdots, A_n$ be a partition of the sample space $\varOmega$. Then for any event $B$
    \begin{equation}
    \begin{aligned}
        \mathbf{P}(A_i | B) &= \frac{\mathbf{P}(A_i \cap B)}{\mathbf{P}(B)} \\
        &= \frac{\mathbf{P}(A_i) \cdot \mathbf{P}(B | A_i)}{\sum_{j=1}^{n} \mathbf{P}(A_j) \cdot \mathbf{P}(B | A_j)} \\
        &= \frac{\mathbf{P}(A_i) \cdot \mathbf{P}(B | A_i)}{\sum_{j=1}^{n} \mathbf{P}(A_j) \cdot \mathbf{P}(B | A_j)}
    \end{aligned}
    \end{equation}
\end{theorem}

A: cause, B: result. Bayes' Theorem is used to calculate the cause given the result. \textit{(inference based on probability)}

\chapter{Independence}
\section{Independence}
\begin{definition}[Independence]
    Event $A$ and $B$ are independent if:
    \begin{equation}
        \mathbf{P}(A \cap B) = \mathbf{P}(A) \cdot \mathbf{P}(B)
    \end{equation}
    or equivalently, when $\mathbf{P}(B) \neq 0$:
    \begin{equation}
        \mathbf{P}(A|B) = \mathbf{P}(A)
    \end{equation}
\end{definition}
Occurrence of $B$ does not provides no information about the occurrence of $A$.
\begin{remark}
    Disjoint events are not independent.
    \begin{equation}
    \begin{aligned}
        \mathbf{P}(A \cap B) &= 0 \\
        \mathbf{P}(A \cap B) &= \mathbf{P}(A) \cdot \mathbf{P}(B)
    \end{aligned}
    \end{equation}
    Only if $\mathbf{P}(A) = 0$ or $\mathbf{P}(B) = 0$, the two equations can be both satisfied. E.g. when a student is in one class, it is unlikely that he is in another class.
\end{remark}

\section{Conditional Independence}
\begin{definition}[Conditional Independence]
    Event $A$ and $B$ are conditionally independent given $C$ if:
    \begin{equation}
        \mathbf{P}(A \cap B | C) = \mathbf{P}(A | C) \cdot \mathbf{P}(B | C)
    \end{equation}
\end{definition}
\begin{remark}
    Conditional independence does not imply independence. Information of the mainland of China does not imply the information of Tsinghua University.
\begin{figure}[H]
    \centering
    \includegraphics[width=0.3\textwidth]{images/Note-3.1.png}
    \caption{Conditional Independence}
    \label{fig:conditional-independence}
\end{figure}
As shown in Figure \ref{fig:conditional-independence}, assume $A$ and $B$ are independent. Obviously, $A$ and $B$ are not independent given $C$ (they are disjoint in $C$).
\end{remark}

\section{Independence of Multiple Events}
\begin{definition}[Independence of multiple events]
    Events $A_1, A_2, \cdots, A_n$ are independent if
    \begin{equation}
        \mathbf{P}(A_{i_1} \cap A_{i_2} \cap \cdots \cap A_{i_k}) = \mathbf{P}(A_{i_1}) \cdot \mathbf{P}(A_{i_2}) \cdots \mathbf{P}(A_{i_k})
    \end{equation}
    for any subset $\{i_1, i_2, \cdots, i_k\} \subseteq \{A_1, A_2, \cdots, A_n\}$.
\end{definition}
\begin{remark}
    Pair independence does not imply independence of multiple events. We should check if $\mathbf{P}(\bigcap_{i=1}^{n} A_i) = \prod_{i=1}^{n} \mathbf{P}(A_i)$.
\end{remark}

\chapter{Discrete Random Variables}
\section{Random Variables}
Random variables (RV) is not a variable, but a diterministic function that maps the sample space to the real number, whose range is either finite or countably infinite. Notation: $X(\omega) = x$.
\begin{definition}[Random Variable]
    Given a probability space $(\varOmega, \mathcal{F}, \mathbf{P})$, a random variable $X$ is a function $X: \varOmega \rightarrow \mathbb{R}$ such that for any $x \in \mathbb{R}$, the set $\{\omega \in \varOmega: X(\omega) \leq x\} \in \mathcal{F}$.
\end{definition}
The intuition of introducing RV is to map the sample space to $\mathbb{R}$ so that we can apply some mathmatical tools to analyze the probability distribution of the sample space. But the probability meature on $\mathbb{R}$ is defined on Borel field $\mathcal{B}(\mathbb{R})$, \textbf{so if we want to calculate the probability of RV}, we should consider the a set $A \in \mathcal{B}(\mathbb{R})$. To maintain the consistency between RV and events, the pre-image of $A$ under $X$ should be in $\mathcal{F}$, i.e. for any $x \in \mathbb{R}$, the set $\{\omega \in \varOmega: X(\omega) \leq x\} \in \mathcal{F}$.
\begin{remark}
    The requirement $\forall x \in \mathbb{R},~ \{\omega \in \varOmega: X(\omega) \leq x\} \in \mathcal{F}$ is naturally met when the events in $\varOmega$ are discrete, because we can choose $\mathcal{F} = 2^{\varOmega}$.
\end{remark}
\begin{example}[Examples of RV] ~ \\ 
    A sequence of 5 tosses of a coin:
    \begin{itemize}
        \item The number of heads in the sequence [Discrete RV]
        \item The 5-long sequence is not a RV.
    \end{itemize}
    Two rolls of a die:
    \begin{itemize}
        \item The sum of two rolls [Discrete RV]
        \item The second roll to the power of the first roll [Discrete RV]
    \end{itemize}
    DeepSeek's inference:
    \begin{itemize}
        \item The latency in generating a response [Continuous RV]
        \item The number of requests in a second [Discrete RV]
    \end{itemize}
\end{example}

\section{Probability Mass Function}
\begin{definition}[Probability Mass Function]
    Probability mass function (PMF) is the ``probability law'' or ``probability distribution'' of a \textbf{discrete RV} $X$.
    \begin{align}
        p_{X}(x) &= \mathbf{P}(X = x) = \\ 
                 &= \mathbf{P}(\{\omega \in \varOmega: X(\omega) = x\})
    \end{align}
\end{definition}
Several properties of PMF can be derived from the axioms of probability, e.g. $p_{X}(x) \geq 0$, $\sum_{x} p_{X}(x) = 1$.
\begin{example}[Binomial PMF]
    \begin{align}
        p_{X}(k) &= \binom{n}{k} p^{k} (1-p)^{n-k}
    \end{align}
    One example: the number of heads in $n$ independent tosses of a coin.
\end{example}
\begin{example}[Geometric PMF]
    \begin{align}
        p_{X}(k) &= (1-p)^{k-1} p
    \end{align}
    One example: the number of tosses until the first head appears.
\end{example}
\begin{example}[Poisson PMF]
    \begin{align}
        p_{X}(k) = \mathrm{e}^{-\lambda} \frac{\lambda^{k}}{k!}
    \end{align}
    It's a good approximation of the binomial distribution when $n$ is large and $p$ is small.
    \begin{proof}
        Let $p = \frac{\lambda}{n}$, then
        \begin{align}
            \binom{n}{k} p^{k} (1-p)^{n-k} &= \frac{n!}{k!(n-k)!} p^{k} (1-p)^{n-k} \\
            &= \frac{n(n-1)\cdots(n-k+1)}{k!} p^{k} (1-p)^{n-k} \\
            &\approx \frac{n^{k}}{k!} p^{k} (1-p)^{n-k} \\
            &= \frac{n^{k}}{k!} \left(\frac{\lambda}{n}\right)^{k} \left(1-\frac{\lambda}{n}\right)^{n} \\
            &\approx \frac{n^{k}}{k!} \left(\frac{\lambda}{n}\right)^{k} \mathrm{e}^{-\lambda} \\
            &= \mathrm{e}^{-\lambda} \frac{\lambda^{k}}{k!}
        \end{align}
    \end{proof}
\end{example}
Note that all these PMFs are normalized.

\section{Functions of Random Variables}
\begin{definition}[Function of Random Variables]
    Given a RV $X$, a function $Y = g(X)$ is also a RV. If $X$ is discrete, then $Y$ is also discrete, with PMF $p_{Y}$ given by
    \begin{align}
        p_{Y}(y) = \sum_{x: g(x) = y} p_{X}(x)
    \end{align}
\end{definition}
\begin{remark}
    The function $g$ is defined on $\mathbb{R}$, i.e. $g: \mathbb{R} \rightarrow \mathbb{R}$. The meaning of $Y = g(X)$ in fact is $Y(\omega) = g(X(\omega))$.
\end{remark}

\section{Expectation and Variance}
\begin{definition}[Expectation of Random Variables]
    The expected value (also called the expectation or the mean) of a random variable $X$, with PMF $p_{X}$, is defined as
    \begin{align}
        \mathbf{E}[X] = \sum_{x} x p_{X}(x)
    \end{align}
\end{definition}
While the PMF gives us the full information about the distribution of $X$, sometimes we still need an overall measure of $X$.
\begin{property}[Properties of Expectation] ~ 
    \begin{itemize}
        \item Let $X$ be a RV and $Y = g(X)$. Sometimes $\mathbf{E}[Y] = \sum_{y} y p_{Y}(y)$ is hard to calculate. Instead, we can use $\mathbf{E}[Y] = \sum_{x} g(x) p_{X}(x)$ which is easier.
        \begin{proof}
            \begin{align}
                \mathbf{E}[Y] &= \sum_{y} y p_{Y}(y) \\
                              &= \sum_{y} y \sum_{x: g(x) = y} p_{X}(x) \\
                              &= \sum_{x} g(x) p_{X}(x)
            \end{align}
        \end{proof}
        \item In general, $\mathbf{E}[g(X)] \neq g(\mathbf{E}[X])$.
        \item If $\alpha_1, \alpha_2, \beta$ are constants, $X, Y$ are RVs, then
        \begin{align}
            \mathbf{E}[\alpha_1 X + \alpha_2 Y + \beta] = \alpha_1 \mathbf{E}[X] + \alpha_2 \mathbf{E}[Y] + \beta
        \end{align}
    \end{itemize}
\end{property}
\begin{definition}[Variance of Random Variables]
    The variance of a RV $X$ is defined as
    \begin{align}
        \mathrm{var}(X) = \mathbf{E}[(X - \mathbf{E}[X])^{2}]
    \end{align}
\end{definition}
Remember $\mathbf{E}[g(x)] = \sum_{x} g(x) p_{X}(x)$, then $\mathrm{var}(X) = \sum_{x} (x - \mathbf{E}[X])^{2} p_{X}(x)$.
We can also define the standard deviation as $\sigma(X) = \sqrt{\mathrm{var}(X)}$.
\begin{property}[Properties of Variance] ~ 
    \begin{itemize}
        \item Define second moment as $\mathbf{E}[X^2] = \sum_x x^2 p_{X}(x)$, then $\mathrm{var}(X)$ can be written in terms of moments expression $\mathrm{var}(X) = \mathbf{E}[X^{2}] - \mathbf{E}[X]^{2}$.
        \item $\mathrm{var}(X) \geq 0$.
        \item $\mathrm{var}(\alpha X + \beta) = \alpha^{2} \mathrm{var}(X)$.
    \end{itemize}
\end{property}
\begin{example}[Expectation and Variance of Binomial Distribution]
    \begin{align}
    &\begin{aligned}
        \mathbf{E}[X] &= \sum_{k=0}^{n} k \binom{n}{k} p^{k} (1-p)^{n-k} \\
                      &= \sum_{k=1}^{n} np \binom{n-1}{k-1} p^{k-1} (1-p)^{n-k} \\
                      &= np \sum_{k=1}^{n} \binom{n-1}{k-1} p^{k-1} (1-p)^{n-k} \\
                      &= np \sum_{k=0}^{n-1} \binom{n-1}{k} p^{k} (1-p)^{n-1-k} \\
                      &= np \cdot \mathrm{PMF} = np
    \end{aligned} \\
    &\begin{aligned}
        \mathbf{E}[X^{2}] =& \sum_{k=0}^{n} k^{2} \binom{n}{k} p^{k} (1-p)^{n-k} \\
                         =& np \sum_{k=1}^{n} k \binom{n-1}{k-1} p^{k-1} (1-p)^{n-k} \\
                         =& np \sum_{k=1}^{n} (k-1+1) \binom{n-1}{k-1} p^{k-1} (1-p)^{n-k} \\ 
                         =& n(n-1)p^2 \sum_{k=2}^{n} \binom{n-2}{k-2} p^{k-2} (1-p)^{n-k} \\ 
                         &+ np \sum_{k=1}^{n} (k-1+1) \binom{n-1}{k-1} p^{k-1} (1-p)^{n-k} \\ 
                         =& n(n-1)p^2 + np
    \end{aligned} \\
    &\begin{aligned}
        \mathrm{var}(X) &= \mathbf{E}[X^{2}] - \mathbf{E}[X]^{2} \\
                        &= n(n-1)p^2 + np - n^2p^2 \\
                        &= np(1-p)
    \end{aligned}
    \end{align}
    Let $n = 1$, then the binomial distribution is the Bernoulli distribution.
\end{example}
\begin{example}[Expectation and Variance of Poisson Distribution]
    \begin{align}
    &\begin{aligned}
        \mathbf{E}[X] &= \sum_{k=0}^{\infty} k \mathrm{e}^{-\lambda} \frac{\lambda^{k}}{k!} \\
                      &= \lambda \mathrm{e}^{-\lambda} \sum_{k=1}^{\infty} \frac{\lambda^{k-1}}{(k-1)!} \\
                      &= \lambda \mathrm{e}^{-\lambda} \mathrm{e}^{\lambda} \\
                      &= \lambda
    \end{aligned} \\
    &\begin{aligned}
        \mathbf{E}[X^{2}] &= \sum_{k=0}^{\infty} k^{2} \mathrm{e}^{-\lambda} \frac{\lambda^{k}}{k!} \\
                         &= \lambda \mathrm{e}^{-\lambda} \sum_{k=1}^{\infty} k \frac{\lambda^{k-1}}{(k-1)!} \\
                         &= \lambda \mathrm{e}^{-\lambda} \sum_{k=1}^{\infty} (k-1+1) \frac{\lambda^{k-1}}{(k-1)!} \\
                         &= \lambda \mathrm{e}^{-\lambda} \left(\lambda \sum_{k=1}^{\infty} \frac{\lambda^{k-1}}{(k-1)!} + \sum_{k=1}^{\infty} \frac{\lambda^{k-1}}{(k-1)!}\right) \\
                         &= \lambda \mathrm{e}^{-\lambda} (\lambda \mathrm{e}^{\lambda} + \mathrm{e}^{\lambda}) \\
                         &= \lambda^{2} + \lambda
    \end{aligned} \\
    &\begin{aligned}
        \mathrm{var}(X) &= \mathbf{E}[X^{2}] - \mathbf{E}[X]^{2} \\
                        &= \lambda^{2} + \lambda - \lambda^{2} \\
                        &= \lambda
    \end{aligned}
    \end{align}
\end{example}

\chapter{Multiple Random Variables}
\section{Conditioning}
\begin{definition}[Conditional PMF]
    The conditional PMF of $X$ given $A$ is defined as
    \begin{align}
        p_{X|A}(x) = \mathbf{P}(X = x | A) = \frac{\mathbf{P}(\{X = x\} \cap A)}{\mathbf{P}(A)}
    \end{align}
    Specifically, if given $Y = y$, then
    \begin{align}
        p_{X|Y = y}(x) = \mathbf{P}(X = x | Y = y) = \frac{\mathbf{P}(\{X = x\} \cap \{Y = y\})}{\mathbf{P}(Y = y)}
    \end{align}
\end{definition}
Since $\mathbf{P}(A) = \sum_{x} \mathbf{P}(\{X = x\} \cap A)$, we have $\sum_{x} p_{X|A}(x) = 1$.
The conditional PMF limits the sample space to $A$.
\begin{definition}[Conditional Expectation]
    The conditional expectation of $X$ given $A$ is defined as
    \begin{align}
        \mathbf{E}[X|A] = \sum_{x} x p_{X|A}(x)
    \end{align}
    Specifically, if given $Y = y$, then
    \begin{align}
        \mathbf{E}[X|Y = y] = \sum_{x} x p_{X|Y}(x|y)
    \end{align}
\end{definition}
\begin{theorem}[Total Expectation Theorem]
    Partition the sample space into disjoint events $A_{1}, A_{2}, \cdots$, then
    \begin{equation}
    \begin{aligned}
        \mathbf{P}(B) &= \sum_{i} \mathbf{P}(A_{i}) \mathbf{P}(B|A_{i}) \\ 
        p_{X}(x) &= \sum_{i} \mathbf{P}(A_{i}) p_{X|A_{i}}(x) \\
        \mathbf{E}[X] &= \sum_{i} \mathbf{P}(A_{i}) \mathbf{E}[X|A_{i}]
    \end{aligned}
    \end{equation}
    From line 2 to line 3, we multiply $x$ on both sides and sum over $x$.
    \begin{figure}[H]
        \centering
        \includegraphics[width=0.3\textwidth]{images/Note-5.1.png}
        \caption{Total Expectation Theorem}
    \end{figure}
\end{theorem}
\begin{example}[Geometric PMF 1]
    Toss a fair coin independently until a head occurs. Let $X$ be the number of tosses. What is the PMF, expectation, and variance of $X$? 
    \begin{solution}
        \begin{equation}
        \begin{aligned}
            \mathbf{E}[X] &= \sum_{k=1}^{\infty} k \mathbf{P}(X = k) = \sum_{k=1}^{\infty} k (1-p)^{k-1} p = \frac{1}{p} \\ 
            \text{var}(X) &= \mathbf{E}[X^{2}] - \mathbf{E}[X]^{2} = \frac{1-p}{p^{2}}
        \end{aligned}
        \end{equation}
    \end{solution}
\end{example}
\begin{property}[Memoryless Property]
    Given that $X > 2$, the random variable $X - 2$ has same geometric PMF with $X$ (not given that $X > 2$). 
    \begin{equation}
    \begin{aligned}
        p_{X|X > 2}(x) &= \frac{\mathbf{P}(\{X > 2\} \cap \{X = x\})}{\mathbf{P}(X > 2)} \\ 
        &= \frac{(1 - p)^{k-1}p}{1 - p - p(1-p)} = (1-p)^{k-3}p
    \end{aligned}
    \end{equation}
    If we \textit{shift} $k$ by 2, then the PMF is the same. That is to say, the random variable $X - 2$ given $X > 2$ has the same geometric PMF with $X$, and thus $\mathbf{E}[(X - 2)|X > 2] = \mathbf{E}[X]$.
\end{property}
\begin{example}[Geometric PMF 2]
    Toss a fair coin independently until a head occurs. Let $X$ be the number of tosses. What is the PMF, expectation, and variance of $X$? 
    \begin{solution}
        If we use the memoryless property, then 
        \begin{equation}
        \begin{aligned}
            \mathbf{E}[X] &= \mathbf{P}(X = 1)\mathbf{E}[X | X = 1] + \mathbf{P}(X > 1)\mathbf{E}[X | X > 1] \\ 
            &= p + (1-p)\mathbf{E}[(X - 1 + 1) | X > 1] \\
            &= p + (1-p)(1 + \mathbf{E}[X]) \\ 
            &\Rightarrow \mathbf{E}[X] = \frac{1}{p}
        \end{aligned}
        \end{equation}
        Similarly
        \begin{equation}
        \begin{aligned}
            \mathbf{E}[X^2] &= \mathbf{P}(X = 1)\mathbf{E}[X^2 | X = 1] + \mathbf{P}(X > 1)\mathbf{E}[X^2 | X > 1] \\
            &= p + (1-p)\mathbf{E}[(X - 1 + 1)^2 | X > 1] \\
            &= p + (1-p)(1 + 2\mathbf{E}[X] + \mathbf{E}[X^2]) \\
            &\Rightarrow \mathbf{E}[X^2] = \frac{2-p}{p^2}
        \end{aligned}
        \end{equation}
        Then
        \begin{equation}
            \text{var}(X) = \mathbf{E}[X^2] - \mathbf{E}[X]^2 = \frac{1-p}{p^2}
        \end{equation}
    \end{solution}
\end{example}

\section{Multiple Discrete Random Variables}
\begin{definition}[Joint PMF]
    The joint PMF of $X$ and $Y$ is defined as
    \begin{align}
        p_{X, Y}(x, y) = \mathbf{P}(\{X = x\} \cap \{Y = y\})
    \end{align}
\end{definition}
\begin{property}[Properties of Joint PMF] ~
    \begin{itemize}
        \item Added up to 1: $\sum_x \sum_y p_{X, Y}(x, y) = 1$.
        \item Marginal PMF: $p_{X}(x) = \sum_y p_{X, Y}(x, y)$, $p_{Y}(y) = \sum_x p_{X, Y}(x, y)$. (Total Probability Theorem)
        \item Conditional PMF: $p_{X|Y}(x|y) = \frac{p_{X, Y}(x, y)}{p_{Y}(y)}$, $p_{Y|X}(y|x) = \frac{p_{X, Y}(x, y)}{p_{X}(x)}$.
    \end{itemize}
\end{property}
If we have a RV function of multiple RVs: $Z = g(X, Y)$, then the PMF of $Z$ is
\begin{align}
    p_{Z}(z) = \sum_{x, y: g(x, y) = z} p_{X, Y}(x, y)
\end{align}
\begin{definition}[Expectation of Multiple Random Variables]
    Recall that 
    \begin{align*}
        \mathbf{E}[g(X)] = \sum_{x} g(x) p_{X}(x)
    \end{align*}
    Then the expectation of $g(X, Y)$ is
    \begin{align}
        \mathbf{E}[g(X, Y)] = \sum_{x, y} g(x, y) p_{X, Y}(x, y)
    \end{align}
\end{definition}
\begin{property}[Properties of Expectation] ~
    \begin{itemize}
        \item In general, $\mathbf{E}[X + Y] \neq \mathbf{E}[X] + \mathbf{E}[Y]$.
        \item $\mathbf{E}[\alpha X + \beta Y + \gamma] = \alpha \mathbf{E}[X] + \beta \mathbf{E}[Y] + \gamma$. \textbf{This property holds for any RVs, despite of the independence.} (go back to the definition of expectation and you will see the linearity, which is also the reason of property 1.)
    \end{itemize}
\end{property}
\begin{example}[Coin tossing]
    Toss a coin independently. The coin may not be fair. 
    \begin{solution}
        \begin{equation}
            X_i = \begin{cases}
                1, & \text{if the $i$-th toss is head} \\ 
                0, & \text{otherwise}
            \end{cases}
        \end{equation}
        The sum $X = \sum_{i=1}^n X_i$. Then 
        \begin{equation}
            \mathbf{E}[X] = \sum_{i=1}^n \mathbf{E}[X_i] = np
        \end{equation}
        \begin{equation}
        \begin{aligned}
            \text{var}(X) &= \mathbf{E}[X^2] - \mathbf{E}[X]^2 = (\sum_{i=1}^{N} X_i^2 + \sum_{i \neq j} X_i X_j) - n^2 p^2 \\
            &= np + n(n-1)p^2 - n^2 p^2 = np(1-p)
        \end{aligned}
        \end{equation}
    \end{solution}
\end{example}

\section{Independence}
\begin{definition}[Definition of Independence]
    Two RVs $X$ and $Y$ are independent if
    \begin{align}
        p_{X, Y}(x, y) = p_{X}(x) \cdot p_{Y}(y)
    \end{align}
    for all $x$ and $y$.
\end{definition}
\begin{property}[Properties of Independence] ~
    \begin{itemize}
        \item $\mathbf{E}[g(X)h(Y)] = \mathbf{E}[g(X)] \cdot \mathbf{E}[h(Y)]$. (go back to the definition of expectation, divide the sum into two parts, and you will see the independence.)
        \item $\text{var}(g(X) + h(Y)) = \text{var}(g(X)) + \text{var}(h(Y))$. (use $\text{var}(A + B) = \mathbf{E}[(A + B)^2] - \mathbf{E}[A + B]^2$.)
    \end{itemize}
\end{property}

\begin{definition}[Definition of Conditional Independence]
    Two RVs $X$ and $Y$ are conditionally independent given $A$ if
    \begin{align}
        p_{X, Y|A}(x, y) = p_{X|A}(x) p_{Y|A}(y)
    \end{align}
    for all $x$ and $y$.
\end{definition}
\begin{example}[Data packet problem]
    A network system with n nodes randomly redistributes each node's data packet through a central processor. Let $X$ be the number of nodes that receive their original data packets. Find $\mathbf{E}[X]$ and $\text{var}(X)$. 
    \begin{solution}
        Define $X_i$ as the indicator of the $i$-th node. Then $X = \sum_{i=1}^n X_i$. 
        \begin{equation}
            \mathbf{E}[X] = \sum_{i=1}^{n} \mathbf{E}[X_i] = n \cdot \frac{1}{n} = 1
        \end{equation}
        \begin{equation}
        \begin{aligned}
            \text{var}(X) &= \mathbf{E}[X^2] - \mathbf{E}[X]^2 = \sum_{i=1}^{n} \mathbf{E}[X_i^2] + \sum_{i \neq j} \mathbf{E}[X_i X_j] - 1 \\ 
            &= n \cdot \frac{1}{n} + \sum_{i=1}^{n} \left(\mathbf{P}(X_i = 1) \cdot \mathbf{P}(X_j = 1 | X_i = 1)\right) - 1 \\ 
            &= n(n-1) \cdot \frac{1}{n(n-1)} = 1
        \end{aligned}
        \end{equation}
    \end{solution}
\end{example}

\chapter{Continuous Random Variables}


\section{Continuous RVs and PDFs}

\begin{definition}[Continuous Random Variables]
    A RV $X$ is continuous if there exists a nonnegative function $f_{X}(x)$, called \textbf{probability density function}, such that
    \begin{align}
        \mathbf{P}(X \in A) = \int_{A} f_{X}(x) \dd{x}
    \end{align}
    Especially, if $A \subseteq \mathcal{B}(\mathbb{R})$, then
    \begin{align}
        \mathbf{P}(a \leq X \leq b) = \int_{a}^{b} f_{X}(x) \dd{x}
    \end{align}
\end{definition}

\begin{property}[Properties of PDF]
    Let $X$ be a continuous RV with PDF $f_{X}(x)$, then
    \begin{itemize}
        \item $f_{X}(x) \geq 0$, and $f_{X}(x)$ can be any nonnegative number. (even $+\infty$)
        \item $\int_{-\infty}^{\infty} f_{X}(x) \dd{x} = 1$.
        \item $\mathbf{P}(x \leq X \leq x + \delta) = \int_{x}^{x + \delta} f_{X}(x) \dd{x} \approx f_{X}(x) \cdot \delta$.
    \end{itemize}
\end{property}


\section{Expectation and Variance}

\begin{definition}[Definitions of Expectation and Variance]
    Let $X$ be a continuous RV with PDF $f_{X}(x)$, then
    \begin{equation}
    \begin{aligned}
        &\mathbf{E}[X] = \int_{-\infty}^{\infty} x f_{X}(x) \dd{x} \\ 
        &\mathbf{E}[g(X)] = \int_{-\infty}^{\infty} g(x) f_{X}(x) \dd{x}
    \end{aligned}
    \end{equation}
    \begin{equation}
    \begin{aligned}
        \text{var}(X) &= \mathbf{E}[(X - \mathbf{E}[X])^{2}] = \int_{-\infty}^{\infty} (x - \mathbf{E}[X])^{2} f_{X}(x) \dd{x} \\ 
        &= \mathbf{E}[X^{2}] - \mathbf{E}[X]^{2}
    \end{aligned}
    \end{equation}
\end{definition}

\begin{example}[Uniform RV]
    Consider a RV $X$ that takes value in an interval $[a, b]$ with PDF
    \begin{figure}[H]
        \centering
        \includegraphics[width=0.3\textwidth]{images/Note-6.1.png}
    \end{figure}
    \vspace{-2em}
    \begin{equation}
        \mathbf{E}[X] = \int_{a}^{b} \frac{x}{b - a} \dd{x} = \frac{a + b}{2}
    \end{equation}
    \begin{equation}
        \text{var}(X) = \mathbf{E}[X^2] - (\mathbf{E}[X])^2 = \int_{a}^{b} \frac{x^2}{b - a} \dd{x} - \left(\frac{a + b}{2}\right)^2 = \frac{(b - a)^2}{12}
    \end{equation}
\end{example}

\begin{example}[Exponential RV]
    Consider an exponential RV $X$ that has a PDF of the form
    \begin{equation}
        f_{X}(x) = \lambda \mathrm{e}^{-\lambda x},~ x \geq 0
    \end{equation}
    \begin{equation}
        \mathbf{E}[X] = \int_{0}^{\infty} x \lambda \mathrm{e}^{-\lambda x} \dd{x} = (-x \mathrm{e}^{-\lambda x}) \Big|_{0}^{\infty} + \int_{0}^{\infty} \mathrm{e}^{-\lambda x} \dd{x} = \frac{1}{\lambda}
    \end{equation}
    \begin{equation}
        \text{var}(X) = \mathbf{E}[X^2] - (\mathbf{E}[X])^2 = (-x^2 \mathrm{e}^{-\lambda x}) \Big|_{0}^{\infty} + \int_{0}^{\infty} 2x \mathrm{e}^{-\lambda x} \dd{x} - \frac{1}{\lambda^2} = \frac{1}{\lambda^2}
    \end{equation}
\end{example}


\section{Comulative Distribution Function}
We want to describe both discrete and continuous RVs in a unified way. An intuition is to ``accumulate'' the probability ``up to'' the value $x$.

\begin{definition}[CDF]
    The cumulative distribution function (CDF) of a RV $X$ is defined as
    \begin{align}
        F_{X}(x) = \mathbf{P}(X \leq x) = \left\{
        \begin{aligned}
            &\sum_{t \leq x} p_{X}(t), &\text{discrete} \\ 
            &\int_{-\infty}^{x} f_{X}(t) \dd{t}, &\text{continuous}
        \end{aligned}
        \right.
    \end{align}
\end{definition}

\begin{example}[Some examples of CDF] ~
    \begin{figure}[H]
        \centering
        \includegraphics[width=0.6\textwidth]{images/Note-6.2.png}
        \caption{CDF: Continuous RV}
    \end{figure}
    \begin{figure}[H]
        \centering
        \includegraphics[width=0.6\textwidth]{images/Note-6.3.png}
        \caption{CDF: Discrete RV}
    \end{figure}
\end{example}

\begin{property}[Properties of CDF] ~
    \begin{itemize}
        \item Monotonically nondecreasing: $x \leq y \Rightarrow F_{X}(x) \leq F_{X}(y)$. (since $p_{X}(x) \geq 0$ and $f_{X}(x) \geq 0$)
        \item $F_X(x) \to 0$ as $x \to -\infty$ and $F_X(x) \to 1$ as $x \to \infty$.
        \item When $X$ is continuous: $F_{X}(x)$ is a continuous function, $f_{X}(x) = \dv{F_X}{x}$.
        \item When $X$ is discrete: $F_{X}(x)$ i piecewise constant function, $p_{X}(k) = F_{X}(k) - F_{X}(k-1)$.
    \end{itemize}
\end{property}

\begin{example}[CDF of Geometry RV]
    Let $X$ be a geometry RV with parameter $p$, then
    \begin{equation}
    \begin{aligned}
        &\mathbf{P}(X = k) = p(1 - p)^{k - 1} \\ 
        \Rightarrow &F_{X}(x) = \sum_{k \leq x} p(1 - p)^{k - 1} = 1 - (1 - p)^{\lfloor x \rfloor}
    \end{aligned}
    \end{equation}
\end{example}

\begin{example}[Maximum test score]
    Assume that each test takes one of the values from 1 to 10 with equal probability 1/10 independently. What is the PMF of the maximum of three test scores?
    \begin{equation}
        X = \max\{X_1, X_2, X_3\}
    \end{equation}
    \begin{solution}
        We have the CDF of $X$ as
        \begin{equation}
        \begin{aligned}
            F_{X}(x) &= \mathbf{P}(X \leq x) \\ 
            &= \mathbf{P}(\max\{X_1, X_2, X_3\} \leq x) \\ 
            &= \mathbf{P}(X_1 \leq x, X_2 \leq x, X_3 \leq x) \\ 
            &= \mathbf{P}(X_1 \leq x) \cdot \mathbf{P}(X_2 \leq x) \cdot \mathbf{P}(X_3 \leq x) \\
            &= F_{X_1}(x) \cdot F_{X_2}(x) \cdot F_{X_3}(x) \\ 
            &= \left(\frac{\lfloor x \rfloor}{10}\right)^3
        \end{aligned}
        \end{equation}
        and the PMF of $X$ is
        \begin{equation}
            p_{X}(x) = F_{X}(x) - F_{X}(x - 1) = \left(\frac{\lfloor x \rfloor}{10}\right)^3 - \left(\frac{\lfloor x \rfloor - 1}{10}\right)^3
        \end{equation}
        Similarly, this approach can be introduced to the minimum of three test scores (need to reverse by $1 - \mathbf{P}(xxx)$).
    \end{solution}
\end{example}


\section{Normal and Exponential RVs}

\begin{definition}[Gaussian (Normal) RV] ~
    \begin{itemize}
        \item Standard normal $N(0, 1)$:
        \begin{equation}
            f_{X}(x) = \frac{1}{\sqrt{2\pi}} \mathrm{e}^{-x^2/2}
        \end{equation}
        which has $\mathbf{E}[X] = 0$ and $\text{var}(X) = 1$. (intergrate by part)
        \item General normal $N(\mu, \sigma^2)$:
        \begin{equation}
            f_{X}(x) = \frac{1}{\sqrt{2\pi} \sigma} \mathrm{e}^{-(x - \mu)^2/2\sigma^2}
        \end{equation}
        which has $\mathbf{E}[X] = \mu$ and $\text{var}(X) = \sigma^2$. (let $Y = (X - \mu) / \sigma$ and $Y$ is standard normal)
        \item The CDF of standard normal is hard to calculate, we denote it as 
        \begin{equation}
            \Phi(x) = \frac{1}{\sqrt{2\pi}} \int_{-\infty}^{x} \mathrm{e}^{-t^2/2} \dd{t}    
        \end{equation}
        If $X \sim N(\mu, \sigma^2)$, then $\mathbf{P}(X \leq x) = \mathbf{P}(\frac{X - \mu}{\sigma} \leq \frac{x - \mu}{\sigma}) = \Phi(\frac{x - \mu}{\sigma})$.
    \end{itemize}
\end{definition}

\begin{definition}[Exponential RV]
    A RV $X$ is exponential with parameter $\lambda$ if it has PDF
    \begin{equation}
        f_{X}(x) = \lambda \mathrm{e}^{-\lambda x},~ x \geq 0
    \end{equation}
    which has $\mathbf{E}[X] = 1 / \lambda$ and $\text{var}(X) = 1 / \lambda^2$, and the CDF is
    \begin{equation}
        F_{X}(x) = 1 - \int_{x}^{+\infty} \lambda \mathrm{e}^{-\lambda t} \dd{t} = 1 - \mathrm{e}^{-\lambda x}
    \end{equation}
\end{definition}

A continuous exponential RV also has the memoryless property as discrete exponential RV, which means
\begin{property}[Memoryless Property]
    Let $X$ be an exponential RV with parameter $\lambda$, then for all $x, c \geq 0$,
    \begin{equation}
        \mathbf{P}(X > x + c \mid X > c) = \mathbf{P}(X > x) = \mathrm{e}^{-\lambda x}
    \end{equation}
\end{property}

Because $\mathrm{e}^{-\lambda x} \sim x + 1, ~x \to 0$, we have
\begin{property}[Continuous-time analog of the geometric] ~
    \begin{itemize} 
        \item $\mathbf{P}(0 \leq X \leq \delta) = 1 - \mathrm{e}^{-\lambda \delta} \approx \lambda \delta$.
        \item $\mathbf{P}(c \leq X \leq c + \delta \mid X > c) \approx \lambda \delta$.
    \end{itemize}
\end{property}

\chapter{Multiple Continuous Random Variables}


\section{Multiple Continuous Random Variables}

\begin{definition}[Joint PDF]
    The two continuous RVs $X$ and $Y$, with the same experiment, are jointly continuous if they can be described by a joint PDF $f_{X, Y}(x, y)$, where $f_{X, Y}(x, y) \geq 0$ and satisfies
    \begin{equation}
        \mathbf{P}((X, Y) \in A) = \iint_{A} f_{X, Y}(x, y) \dd{x} \dd{y}
    \end{equation}
    Especially, if $A \subseteq \mathcal{B}(\mathbb{R}^2)$, then
    \begin{equation}
        \mathbf{P}((X, Y) \in [a, b] \times [c, d]) = \int_{c}^{d} \int_{a}^{b} f_{X, Y}(x, y) \dd{x} \dd{y}
    \end{equation}
\end{definition}

\begin{property}[Properties of Joint PDFs] ~
    \begin{itemize}
        \item Normalization: $\iint_{\Omega} f_{X, Y}(x, y) \dd{x} \dd{y} = 1$.
        \item Interpretation: $f_{X, Y}(x, y) \dd{x} \dd{y}$ is the probability that $(X, Y)$ lies in the small rectangle $[x, x + \dd{x}] \times [y, y + \dd{y}]$.
        \item Marginal PDFs: $f_{X}(x) = \int_{\Omega_{Y}} f_{X, Y}(x, y) \dd{y}$ and $f_{Y}(y) = \int_{\Omega_{X}} f_{X, Y}(x, y) \dd{x}$.
    \end{itemize}
\end{property}

\begin{definition}[Joint CDFs]
    If $X$ and $Y$ are two continuous RVs associated with the same experiment, then the joint CDF of $X$ and $Y$ is defined as
    \begin{equation}
        F_{X, Y}(x, y) = \mathbf{P}(X \leq x, Y \leq y)
    \end{equation}
    If $X$ and $Y$ can be described by a joint PDF $f_{X, Y}(x, y)$, then
    \begin{equation}
        F_{X, Y}(x, y) = \int_{-\infty}^{x} \int_{-\infty}^{y} f_{X, Y}(t, s) \dd{t} \dd{s}
    \end{equation}
    Conversely, if $F_{X, Y}(x, y)$ is differentiable, then
    \begin{equation}
        f_{X, Y}(x, y) = \pdv{F_{X, Y}(x, y)}{x}{y}
    \end{equation}
\end{definition}

\begin{definition}[Expectation and Variance]
    Let $X$ and $Y$ be two continuous RVs with joint PDF $f_{X, Y}(x, y)$, then
    \begin{equation}
    \begin{aligned}
        &\mathbf{E}[g(X, Y)] = \iint_{\Omega} g(x, y) f_{X, Y}(x, y) \dd{x} \dd{y} \\ 
        &\text{var}(g(X, Y)) = \mathbf{E}[g(X, Y) - \mathbf{E}[g(X, Y)]]^2 = \mathbf{E}[g(X, Y)^2] - \mathbf{E}[g(X, Y)]^2
    \end{aligned}
    \end{equation}
\end{definition}

\begin{definition}[Independence]
    $X$ and $Y$ are independent if their joint PDF can be factorized as
    \begin{equation}
        f_{X, Y}(x, y) = f_{X}(x) f_{Y}(y), ~\forall x, y
    \end{equation}
\end{definition}

\begin{example}[Buffon's needle]
    Parallel lines are at distance $d$, throw a needle of length $l$ (assume $l < d$). Find $\mathbf{P}(\text{needle intersects one of the lines})$.  
    \begin{solution}
        Let $X$ be the distance from the center of the needle to the nearest line, and $\Theta$ be the angle between the needle and the lines, where $X$ and $\Theta$ are independent. Given the symmetry characteristics, it is well to assume $X$ is uniform in $[0, d/2]$ and $\Theta$ is uniform in $[0, \pi/2]$. Then the joint PDF is
    \begin{equation}
        f_{X, \Theta}(x, \theta) = \frac{2}{d} \cdot \frac{2}{\pi}, \quad 0 \leq x \leq \frac{d}{2}, 0 \leq \theta \leq \frac{\pi}{2}
    \end{equation}
    The needle intersects if $X \leq l/2 \sin\Theta$, and the probability is
    \begin{equation}
    \begin{aligned}
        \mathbf{P}(X \leq \frac{l}{2}\sin\Theta) &= \iint_{X\leq\frac{l}{2}\sin\Theta} f_{X, \Theta}(x, \theta) \dd{x} \dd{\theta} \\ 
        &= \frac{4}{\pi d} \int_{0}^{\pi/2} \int_{0}^{(l/2)\sin\theta} \dd{x} \dd{\theta} = \frac{2l}{\pi d}
    \end{aligned}
    \end{equation}
    \end{solution}
\end{example}


\section{Conditioning and Independence}

\begin{definition}[Definition of Conditional PDFs]
    Let $X$ and $Y$ be two continuous RVs with joint PDF $f_{X, Y}(x, y)$, then the conditional PDF of $X$ given $Y = y$ is defined as
    \begin{equation}
        f_{X \mid Y}(x \mid y) = \frac{f_{X, Y}(x, y)}{f_{Y}(y)}
    \end{equation}
\end{definition}

\begin{remark}
    There's a paradox that the denominator $\mathbf{P}(Y = y) = 0$ for continuous RVs, but we can let $Y$ be a little fatter: $\mathbf{P}(y \leq Y \leq y + \delta) \approx f_{Y}(y) \cdot \delta$ and let $\delta \to 0$ after we do division.
\end{remark}

\begin{figure}[H]
    \centering
    \includegraphics[width=0.5\textwidth]{images/Note-7.1.png}
    \caption{Visualization of Conditional PDF}
\end{figure}

\begin{example}[Stick-breaking]
    \label{ex:stick-breaking}
    Break a stick of length $l$ twice: break at $X$: uniform in $[0, l]$; break again at $Y$: uniform in $[0, X]$. Find $\mathbf{E}[Y]$. 
    \begin{solution}
        \begin{align}
            f_{X, Y}(x, y) &= f_{X}(x) f_{Y \mid X}(y \mid x) = \frac{1}{l} \cdot \frac{1}{x}, \quad 0 \leq y \leq x \leq l \\ 
            f_{Y}(y) &= \int_{-\infty}^{+\infty} f_{X, Y}(x, y) \dd{x} = \int_{y}^{l} \frac{1}{l} \cdot \frac{1}{x} \dd{x} = \frac{1}{l} \ln\left(\frac{l}{y}\right), \quad 0 \leq y \leq l \\ 
            \mathbf{E}[Y] &= \int_{-\infty}^{+\infty} y f_{Y}(y) \dd{y} = \int_{0}^{l} y \cdot \frac{1}{l} \ln\left(\frac{l}{y}\right) \dd{y} = \frac{l}{4}
        \end{align}
    \end{solution}
\end{example}

\begin{definition}[Conditional Expectation]
    The conditional expectation of $X$ given that $Y = y$ has happened is defined by
    \begin{equation}
        \mathbf{E}[X \mid Y = y] = \int_{-\infty}^{+\infty} x f_{X \mid Y}(x \mid y) \dd{x}
    \end{equation}
    For a function $g(X, Y)$, the conditional expectation is 
    \begin{equation}
        \mathbf{E}[g(X, Y) \mid Y = y] = \int_{-\infty}^{+\infty} g(x, y) f_{X \mid Y}(x \mid y) \dd{x}
    \end{equation}
\end{definition}

Remember how we prove the total expectation theorem in discrete case: we multiple $x$ on both side of the equation of conditional probability and sum over $x$. Now we can do the same thing in continuous case.
\begin{theorem}[Total Expectation Theorem]
    Let $X$ and $Y$ be two continuous RVs, then
    \begin{equation}
        \mathbf{E}[X] = \mathbf{E}_{Y}\left[\mathbf{E}_{X \mid Y}[X \mid Y]\right] =  \int_{-\infty}^{+\infty} \mathbf{E}[X \mid Y = y] f_{Y}(y) \dd{y}
    \end{equation}
\end{theorem}

\begin{solution}
    \textbf{Another way to solve Example \ref{ex:stick-breaking}.} \\ 
    Conditioning on $X = x$, we have
    \begin{equation}
        \mathbf{E}[Y \mid X = x] = \int_{0}^{x} y f_{Y \mid X}(y \mid x) \dd{y} = \int_{0}^{x} y \cdot \frac{1}{x} \dd{y} = \frac{x}{2}
    \end{equation}
    With total expectation theorem, we have
    \begin{equation}
        \mathbf{E}[Y] = \mathbf{E}_{X}\left[\mathbf{E}[Y \mid X]\right] = \int_{0}^{l} \frac{x}{2} f_{X}(x) \dd{x} = \int_{0}^{l} \frac{x}{2l} \dd{x} = \frac{l}{4}
    \end{equation}
\end{solution}

\begin{definition}[Independence]
    Two continuous RVs $X$ and $Y$ are independent if their joint PDF can be factorized as
    \begin{equation}
        f_{X, Y}(x, y) = f_{X}(x) f_{Y}(y), ~\forall x, y
    \end{equation}
\end{definition}
Similar to discrete case, the following properties hold:
\begin{property}[Properties of Independence]
    \begin{itemize}
        \item $\mathbf{E}[g(X)h(Y)] = \mathbf{E}[g(X)]\mathbf{E}[h(Y)]$ if $X$ and $Y$ are independent.
        \item $\text{var}(X + Y) = \text{var}(X) + \text{var}(Y)$ if $X$ and $Y$ are independent.
    \end{itemize}
\end{property}


\section{Bayes' Theorem}
Change a little bit from $f_{X, Y}(x, y) = f_{X \mid Y}(x \mid y) f_{Y}(y)$, we have
\begin{theorem}[Bayes' Theorem]
    Let $X$ and $Y$ be two continuous RVs, then
    \begin{equation}
        f_{X \mid Y}(x \mid y) = \frac{f_{Y \mid X}(y \mid x) f_{X}(x)}{f_{Y}(y)} = \frac{f_{Y \mid X}(y \mid x) f_{X}(x)}{\int_{-\infty}^{+\infty} f_{Y \mid X}(y \mid t) f_{X}(t) \dd{t}}
    \end{equation}
    which can be interpreted using $f(x)\dd{x} = \mathbf{P}(x \leq X \leq x + \dd{x})$.
\end{theorem}





\chapter{Derived Distributions and Entropy}

\section{Derived Distributions}
\label{sec:derived-distributions}
We've already known the Distribution of a RV $X$ and we want to find the distribution of $Y = g(X)$. In discrete case we can calculate the PMF for each possible value of $Y$ like $\mathbf{P}(Y = y) = \sum_{x: g(x) = y} \mathbf{P}(X = x)$. For continuous case we cannot simply do such thing because the conservation of PDF is not guaranteed under summation. Instead, CDF is standardized and we can use it to find the PDF of $Y$.
\begin{theorem}[Principal Method for Derived Distribution]
    Two-step method to find the distribution of $Y = g(X)$:
    \begin{enumerate}
        \item Find the CDF of $Y$:
        \begin{equation}
            F_{Y}(y) = \mathbf{P}(Y \leq y) = \mathbf{P}(g(X) \leq y)
        \end{equation}
        \item Differentiate the CDF to find the PDF:
        \begin{equation}
            f_{Y}(y) = \dv{F_{Y}}{y}
        \end{equation}
    \end{enumerate}
\end{theorem}

\begin{example}[Taking a Train]
    A professor is taking the high-speed railway from Beijing to Shanghai. Suppose that the speed of the train is uniformly distributed between 280 km/h and 350 km/h. The distance between the two cities is 1400 km. What is the distribution of the duration of the trip?
\end{example}
\begin{solution}
    Let $T(V) = \frac{1400}{V}$ be the duration of the trip. The speed $V$ is uniformly distributed in $[280, 350]$, so the CDF of $V$ is
    \begin{equation}
        F_{V}(v) = \begin{cases}
            0, & v < 280 \\ 
            \frac{v - 280}{350 - 280}, & 280 \leq v < 350 \\ 
            1, & v \geq 350
        \end{cases}
    \end{equation}
    The CDF of $T$ is
    \begin{equation}
        F_{T}(t) = \mathbf{P}(T \leq t) = \mathbf{P}\left(\frac{1400}{V} \leq t\right) = \mathbf{P}\left(V \geq \frac{1400}{t}\right) = 1 - F_{V}\left(\frac{1400}{t}\right)
    \end{equation}
    \begin{equation}
        F_{T}(t) = \begin{cases}
            0, & t < 4 \\ 
            5 - \frac{20}{t}, & 4 \leq t < 5 \\ 
            1, & t \geq 5
        \end{cases}
    \end{equation}
    The PDF of $T$ is
    \begin{equation}
        f_{T}(t) = \dv{F_{T}}{t} = \begin{cases}
            0, & t < 4 \\ 
            \frac{20}{t^2}, & 4 \leq t < 5 \\ 
            0, & t \geq 5
        \end{cases}
    \end{equation}
\end{solution}

\begin{example}[PDF of $Y = aX+b$]
    Let $X$ be a RV with PDF $f_{X}(x)$, and $Y = aX + b$ where $a \neq 0$, then the PDF of $Y$ is
    \begin{equation}
        f_{Y}(y) = \frac{1}{\abs{a}} f_{X}\left(\frac{y - b}{a}\right)
    \end{equation}
\end{example}
\begin{solution}
    The CDF of $Y$ is
    \begin{equation}
        F_{Y}(y) = \mathbf{P}(Y \leq y) = \mathbf{P}(aX + b \leq y) = \mathbf{P}\left(X \leq \frac{y - b}{a}\right) = F_{X}\left(\frac{y - b}{a}\right)
    \end{equation}
    And then differentiating the CDF using chain rule
    \begin{equation}
        f_{Y}(y) = \dv{F_{Y}}{y} = \frac{1}{a} \dv{F_{X}}{\frac{y - b}{a}} = \frac{1}{a} f_{X}\left(\frac{y - b}{a}\right)
    \end{equation}
    If $a < 0$, the inequality will be reversed, and a minus sign will be added to the derivative. So we have
    \begin{equation}
        f_{Y}(y) = \frac{1}{\abs{a}} f_{X}\left(\frac{y - b}{a}\right)
    \end{equation}
\end{solution}

For monotonic functions $g(x)$
\begin{theorem}[PDF of a Strictly Monotonic Function]
    Suppose that $g(\cdot)$ is strictly monotonic, then $g$ has an inverse $h(\cdot) \triangleq g^{-1}(\cdot)$. Assume $h$ is differentiable and the PDF of $Y$ is
    \begin{equation}
        f_{Y}(y) = f_{X}(h(y)) \abs{\dv{h}{y}}
    \end{equation}
\end{theorem}
\begin{proof}
    \begin{align}
        F_{Y}(y) &= \mathbf{P}(Y \leq y) = \mathbf{P}(g(X) \leq y) = \mathbf{P}(X \leq h(y)) = F_{X}(h(y)) \\ 
        f_{Y}(y) &= \dv{F_{Y}}{y} = f_{X}(h(y)) \dv{h}{y}
    \end{align}
    If $g$ is strictly decreasing, the inequality will be reversed, and a minus sign will be added to the derivative. So we have
    \begin{equation}
        f_{Y}(y) = f_{X}(h(y)) \abs{\dv{h}{y}}
    \end{equation}
\end{proof}
\begin{remark}
    If $g$ is a multiple-to-multiple function, $\abs{\dv{h}{y}}$ becomes Jacobian determinant.
\end{remark}

\section{Entropy}
\begin{definition}[Entropy]
    The entropy of a discrete RV $X$ with PMF $p_X(x)$ is defined as
    \begin{equation}
        H(X) = -\sum_{x} p_X(x) \ln p_X(x)
    \end{equation}
    The entropy of a continuous RV $X$ with PDF $f_X(x)$ is defined as
    \begin{equation}
        H(X) = -\int_{-\infty}^{+\infty} f_X(x) \ln f_X(x) \dd{x}
    \end{equation}
\end{definition}
\begin{remark}
    \begin{itemize}
        \item The entropy of a discrete RV is non-negative, while the entropy of a continuous RV can be negative (if $f_{X}(x) > 1$).
        \item The entropy describes the uncertainty of a random experiment.
        \item The base of the logarithm can be different, which is equal to multiplying by a constant. \begin{itemize}
            \item If the base is 2, the unit is bit.
            \item If the base is $e$, the unit is nat.
        \end{itemize}
    \end{itemize}
\end{remark}
\begin{remark} \\ 
    Why is entropy in the form of the expectation of $-\ln f_{X}(x)$? \\ 
    The entropy of a RV is defined as the expectation of the self-information $H(X) = \mathbf{E}[I(p(X))]$. According to Shannon, self-information shoudle satisfies the following axioms:
    \begin{itemize}
        \item Self-information is and only is a function of probability $I(p)$.
        \item If probability $p = p_1 p_2$, then $I(p) = I(p_1) + I(p_2)$.
        \item $I(p)$ is continuous.
    \end{itemize}
    The only function that satisfies these axioms is $I(p) = k \ln p$. We add a negative sign to make it positive, and the constant $k$ is equivalent to the base of the logarithm. 
\end{remark}



\appendix
\chapter{Problem Solutions}


\section{Derived Distributions}
\textbf{In the lecture right after I finished this section, the lecturer provided an ``official'' solution to this problem, see section \ref{sec:derived-distributions}.}

In the homework, I constantly encounter some problems like this:
\begin{example}[PDF of a Function of Multiple Continuous RVs]
    Let $X, Y$ be two continuous RVs with joint PDF $f_{X, Y}(x, y)$, and $Z = g(X, Y)$. Find the PDF of $Z$.
\end{example}

Here I introduce two approaches to solve this kind of problems: CDF and Jacobian transformation.
\begin{solution}
    \textbf{CDF:}
    \begin{enumerate}
        \item Find the CDF of $Z$
            \begin{equation}
                F_{Z}(z) = \mathbf{P}(Z \leq z) = \mathbf{P}(g(X, Y) \leq z) = \iint_{g(x, y) \leq z} f_{X, Y}(x, y) \dd{x} \dd{y}
            \end{equation}
        \item Differentiate $F_{Z}(z)$ to get the PDF of $Z$
            \begin{equation}
                f_{Z}(z) = \dv{F_{Z}(z)}{z}
            \end{equation}
    \end{enumerate}
\end{solution}
\begin{solution}
    \textbf{Jacobian Transformation:} This approach requires finding a valid variable substitution with invertibility. \\ 
    Take $Z = g(X, Y) = X + Y$ as an example. Normally, if we want to calculate the determinant of the Jacobian matrix, the Jacobian matrix must be square. However, here $Z$ is a scalar instead of a vector. We take the following steps to avoid this problem:
    \begin{enumerate}
        \item Let $U = X + Y$ and $V = X$, then $X = V$ and $Y = U - V$.
        \item Calculate the Jacobian matrix and its determinant
            \begin{align}
                &J = \mqty(\pdv{X}{U} & \pdv{X}{V} \\ \pdv{Y}{U} & \pdv{Y}{V}) = \mqty(0 & 1 \\ 1 & -1) \\ 
                &\det(J) = -1 
            \end{align}
        \item Find the joint PDF of $U$ and $V$
            \begin{equation}
                f_{U, V}(u, v) = f_{X, Y}(v, u - v) \cdot \abs{\det(J)} = f_{X, Y}(v, u - v)
            \end{equation}
        \item Integrate out $V$ to get the marginal PDF of $U$, which is the PDF of $Z$
            \begin{equation}
                f_{U}(u) = \int_{-\infty}^{+\infty} f_{X, Y}(v, u - v) \dd{v}
            \end{equation}
    \end{enumerate}
\end{solution}



\end{document}
