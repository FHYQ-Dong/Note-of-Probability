\documentclass[device=normal, lang=en, fontsize=12pt]{elegantnote}
\usepackage{amsmath}
\usepackage{amssymb}
\usepackage{float}
\usepackage{extarrows}
\usepackage{physics}

\definecolor{pgcolor}{RGB}{251, 250, 248}
\pagecolor{pgcolor}
\numberwithin{equation}{section}

\theoremstyle{definition} %
\newtheorem{property}{Property}[section] %

\title{Probability Notes}
\author{FHYQ-Dong}
\date{\today}


\begin{document}

\maketitle
\newpage
\tableofcontents
\newpage


\section{Probability Space}
\begin{definition}[Probability Space]
    Probability space is a tuple $\left(\varOmega,~\mathcal{F},~\mathbf{P}\right)$ where
    \begin{itemize}
        \item Sample space $\varOmega$ is a set of all possible outcomes.
        \item $ \sigma $-algebra $\mathcal{F}$ is a collection of subsets of $\varOmega$.
        \item Probability measure $\mathbf{P}$ is a function that assigns a probability to each event in $\mathcal{F}$.
    \end{itemize}
\end{definition}

\begin{definition}[Sample Space]
    Sample space $\varOmega$ is a set of all possible outcomes, which should satisfy the following properties:
    \begin{itemize}
        \item Mutually Exclusive: The elements in $\varOmega$ should be unique.
        \item Collectively Exhaustive: The elements in $\varOmega$ should cover all possible outcomes.
    \end{itemize}
\end{definition}

Once the experiment is conducted, there is \textbf{exactly one} element in the sample space $\varOmega$ occurs.

\begin{example}[Example of Sample Space]
    Here are some examples of sample space:
    \begin{itemize}
        \item Discrete: $\varOmega = \{1, 2, 3, 4, 5, 6\}$ - Rolling a fair die.
        \item Continuous: $\varOmega = [0, 1]$ - Randomly selecting a real number between 0 and 1.
    \end{itemize}
\end{example}

Before we define the $\sigma$-algebra, we need to introduce the concept of \textbf{countable}. We say a set $X$ is countable if there is a bijection between $X$ and $\mathbb{N}$. Some countable sets are: $\mathbb{N}$, $\mathbb{Z}$, $\mathbb{Q}$, $\mathbb{N}^2$, $\mathbb{Z}^2$, $\mathbb{Q}^2$, $\cup_{n=1}X_{n}$. Some uncountable sets are: $\mathbb{R}$, $\mathbb{C}$, $\mathbb{R}^2$, $\mathbb{C}^2$.

\begin{definition}[$\sigma$-algebra]
    We say a set $\mathcal{F}$  collection of subsets of $\varOmega$ is a $\sigma$-algebra if it satisfies the following properties:
\end{definition}

\begin{remark}
    Not all the things in real worlf can be measured by probability, especially when things are continuous (uncountable).
\end{remark}

\section{Conditional Probability}


\subsection{Conditional Probability}
\begin{definition}[Conditional Probability]
    Conditional probability is the probability of an event $A$ given that another event $B$ has already occurred. It is denoted by $P(A|B)$ and is defined as:
    \begin{equation}
        \mathbf{P}(A|B) = \frac{\mathbf{P}(A \cap B)}{\mathbf{P}(B)}
    \end{equation}
    Note that when $\mathbf{P}(B) = 0$, the conditional probability is undefined.
\end{definition}

We can easily check the conditional probability satisfies the properties of probability measure, which means it is a legitimate probability on a new universe.
\begin{itemize}
    \item Non-negativity: $\mathbf{P}(A|B) \geq 0$.
    \item Normalization: $\mathbf{P}(\varOmega|B) = 1$.
    \item Additivity: when $A$ and $B$ are disjoint, $\mathbf{P}(A \cup B|C) = \frac{\mathbf{P}((A \cup B) \cap C)}{\mathbf{P}(C)} = \frac{\mathbf{P}(A \cap C) + \mathbf{P}(B \cap C)}{\mathbf{P}(C)} = \mathbf{P}(A|C) + \mathbf{P}(B|C)$. (The second equality is due to $A \cap C$ and $B \cap C$ are disjoint).
\end{itemize}

\begin{example}[Discrete and Continuous]
    For discrete case, the conditional probability can be calulated by:
    \begin{equation}
        \mathbf{P}(A|B) = \frac{\#~of~elements~in~A}{\#~of~elements~in~B}
    \end{equation}
    For continuous case, the conditional probability can be calculated by:
    \begin{equation}
        \mathbf{P}(A|B) = \frac{area~of~A}{area~of~B}
    \end{equation}
\end{example}

When solving a problem, the following equations may help:
\begin{align}
    \mathbf{P}(A \cap D) &= \mathbf{P}(A) \cdot \mathbf{P}(D|A) \\
    &= \mathbf{P}(D) \cdot \mathbf{P}(A|D) \\
    \mathbf{P}(A \cap B \cap C) &= \mathbf{P}(A) \cdot \mathbf{P}(B|A) \cdot \mathbf{P}(C|A \cap B)
\end{align}
For the second equation: 1.event $A$ occurs; 2.event $B$ occurs given $A$; 3.event $C$ occurs given $A$ and $B$. Or
\begin{align*}
    \mathbf{P}(A \cap B \cap C) &= \mathbf{P}(A \cap B) \cdot \mathbf{P}(C | A \cap B) \\ 
    &= \mathbf{P}(A) \cdot \mathbf{P}(B | A) \cdot \mathbf{P}(C | A \cap B)
\end{align*}
\begin{remark}
    These equations are not fixed, depending on $\mathbf{P}(A)$, $\mathbf{P(A \cap B)}$ and $\mathbf{P}(A | B)$ which are easier to calculate.
\end{remark}

\subsection{Total Probability Theorem}
We can calculate the total probability with \textit{Devide and Conquer} strategy:
\begin{enumerate}
    \item Devide event $B$ by $A_1, A_2, \cdots A_n$, note that $\bigcup_{i=1}^n A_i = \Omega$ and $A_i \cap A_j = \emptyset$.
    \item Calculate $\mathbf{P}(B | A_i)$ which are easier to calculate.
    \item Sum all the probabilities.
\end{enumerate}
\begin{figure}[H]
    \centering
    \includegraphics[width=0.3\textwidth]{images/image2-1.png}
    \caption{Total Probability Theorem}
    \label{fig:total-probability}
\end{figure}

\begin{theorem}[Total Probability Theorem]
    Let $A_1, A_2, \cdots, A_n$ be a partition of the sample space $\varOmega$. Then for any event $B$,
    \begin{equation}
        \mathbf{P}(B) = \sum_{i=1}^{n} \mathbf{P}(A_i) \cdot \mathbf{P}(B | A_i)
    \end{equation}
\end{theorem}
\begin{example}[Die Rolling]
    You roll a fair four-sided die. If the result is 1 or 2, you roll once more but otherwise, you stop. What is the probability that the sum total of rolls is at least 4?
    \textbf{Solution:}
    \begin{enumerate}
        \item Let $A_i = \{the~first~roll~is~i\}, B = \{the~sum~total~of~rolls~is~at~least~4\}$.
        \item $\mathbf{P}(A_1) = \mathbf{P}(A_2) = \mathbf{P}(A_3) = \mathbf{P}(A_4) = \frac{1}{4}$.
        \item $\mathbf{P}(B | A_1) = \frac{1}{2}, \mathbf{P}(B | A_2) = \frac{3}{4}, \mathbf{P}(B | A_3) = 0, \mathbf{P}(B | A_4) = 1$.
        \item $\mathbf{P}(B) = \frac{9}{16}$.
    \end{enumerate}
\end{example}


\subsection{Bayes' Theorem}
If we know:
\begin{itemize}
    \item ``Prior'' probabilities $\mathbf{P}(A_i)$.
    \item ``Likelihood'' probabilities $\mathbf{P}(B | A_i)$.
\end{itemize}
We wish we can calculate the ``Posterior'' probabilities $\mathbf{P}(A_i | B)$. 

\begin{theorem}[Bayes' Theorem]
    Let $A_1, A_2, \cdots, A_n$ be a partition of the sample space $\varOmega$. Then for any event $B$
    \begin{equation}
    \begin{aligned}
        \mathbf{P}(A_i | B) &= \frac{\mathbf{P}(A_i \cap B)}{\mathbf{P}(B)} \\
        &= \frac{\mathbf{P}(A_i) \cdot \mathbf{P}(B | A_i)}{\sum_{j=1}^{n} \mathbf{P}(A_j) \cdot \mathbf{P}(B | A_j)} \\
        &= \frac{\mathbf{P}(A_i) \cdot \mathbf{P}(B | A_i)}{\sum_{j=1}^{n} \mathbf{P}(A_j) \cdot \mathbf{P}(B | A_j)}
    \end{aligned}
    \end{equation}
\end{theorem}

A: cause, B: result. Bayes' Theorem is used to calculate the cause given the result. \textit{(inference based on probability)}

\section{Independence}
\subsection{Independence}
\begin{definition}[Independence]
    Event $A$ and $B$ are independent if:
    \begin{equation}
        \mathbf{P}(A \cap B) = \mathbf{P}(A) \cdot \mathbf{P}(B)
    \end{equation}
    or equivalently, when $\mathbf{P}(B) \neq 0$:
    \begin{equation}
        \mathbf{P}(A|B) = \mathbf{P}(A)
    \end{equation}
\end{definition}
Occurrence of $B$ does not provides no information about the occurrence of $A$.
\begin{remark}
    Disjoint events are not independent.
    \begin{equation}
    \begin{aligned}
        \mathbf{P}(A \cap B) &= 0 \\
        \mathbf{P}(A \cap B) &= \mathbf{P}(A) \cdot \mathbf{P}(B)
    \end{aligned}
    \end{equation}
    Only if $\mathbf{P}(A) = 0$ or $\mathbf{P}(B) = 0$, the two equations can be both satisfied. E.g. when a student is in one class, it is unlikely that he is in another class.
\end{remark}

\subsection{Conditional Independence}
\begin{definition}[Conditional Independence]
    Event $A$ and $B$ are conditionally independent given $C$ if:
    \begin{equation}
        \mathbf{P}(A \cap B | C) = \mathbf{P}(A | C) \cdot \mathbf{P}(B | C)
    \end{equation}
\end{definition}
\begin{remark}
    Conditional independence does not imply independence. Information of the mainland of China does not imply the information of Tsinghua University.
\begin{figure}[H]
    \centering
    \includegraphics[width=0.3\textwidth]{images/Note-3.1.png}
    \caption{Conditional Independence}
    \label{fig:conditional-independence}
\end{figure}
As shown in Figure \ref{fig:conditional-independence}, assume $A$ and $B$ are independent. Obviously, $A$ and $B$ are not independent given $C$ (they are disjoint in $C$).
\end{remark}

\subsection{Independence of Multiple Events}
\begin{definition}[Independence of multiple events]
    Events $A_1, A_2, \cdots, A_n$ are independent if
    \begin{equation}
        \mathbf{P}(A_{i_1} \cap A_{i_2} \cap \cdots \cap A_{i_k}) = \mathbf{P}(A_{i_1}) \cdot \mathbf{P}(A_{i_2}) \cdots \mathbf{P}(A_{i_k})
    \end{equation}
    for any subset $\{i_1, i_2, \cdots, i_k\} \subseteq \{A_1, A_2, \cdots, A_n\}$.
\end{definition}
\begin{remark}
    Pair independence does not imply independence of multiple events. We should check if $\mathbf{P}(\bigcap_{i=1}^{n} A_i) = \prod_{i=1}^{n} \mathbf{P}(A_i)$.
\end{remark}

\section{Discrete Random Variables}
\subsection{Random Variables}
Random variables (RV) is not a variable, but a diterministic function that maps the sample space to the real number, whose range is either finite or countably infinite. Notation: $X(\omega) = x$.
\begin{definition}[Random Variable]
    Given a probability space $(\varOmega, \mathcal{F}, \mathbf{P})$, a random variable $X$ is a function $X: \varOmega \rightarrow \mathbb{R}$ such that for any $x \in \mathbb{R}$, the set $\{\omega \in \varOmega: X(\omega) \leq x\} \in \mathcal{F}$.
\end{definition}
The intuition of introducing RV is to map the sample space to $\mathbb{R}$ so that we can apply some mathmatical tools to analyze the probability distribution of the sample space. But the probability meature on $\mathbb{R}$ is defined on Borel field $\mathcal{B}(\mathbb{R})$, \textbf{so if we want to calculate the probability of RV}, we should consider the a set $A \in \mathcal{B}(\mathbb{R})$. To maintain the consistency between RV and events, the pre-image of $A$ under $X$ should be in $\mathcal{F}$, i.e. for any $x \in \mathbb{R}$, the set $\{\omega \in \varOmega: X(\omega) \leq x\} \in \mathcal{F}$.
\begin{remark}
    The requirement $\forall x \in \mathbb{R},~ \{\omega \in \varOmega: X(\omega) \leq x\} \in \mathcal{F}$ is naturally met when the events in $\varOmega$ are discrete, because we can choose $\mathcal{F} = 2^{\varOmega}$.
\end{remark}
\begin{example}[Examples of RV] ~ \\ 
    A sequence of 5 tosses of a coin:
    \begin{itemize}
        \item The number of heads in the sequence [Discrete RV]
        \item The 5-long sequence is not a RV.
    \end{itemize}
    Two rolls of a die:
    \begin{itemize}
        \item The sum of two rolls [Discrete RV]
        \item The second roll to the power of the first roll [Discrete RV]
    \end{itemize}
    DeepSeek's inference:
    \begin{itemize}
        \item The latency in generating a response [Continuous RV]
        \item The number of requests in a second [Discrete RV]
    \end{itemize}
\end{example}

\subsection{Probability Mass Function}
\begin{definition}[Probability Mass Function]
    Probability mass function (PMF) is the ``probability law'' or ``probability distribution'' of a \textbf{discrete RV} $X$.
    \begin{align}
        p_{X}(x) &= \mathbf{P}(X = x) = \\ 
                 &= \mathbf{P}(\{\omega \in \varOmega: X(\omega) = x\})
    \end{align}
\end{definition}
Several properties of PMF can be derived from the axioms of probability, e.g. $p_{X}(x) \geq 0$, $\sum_{x} p_{X}(x) = 1$.
\begin{example}[Binomial PMF]
    \begin{align}
        p_{X}(k) &= \binom{n}{k} p^{k} (1-p)^{n-k}
    \end{align}
    One example: the number of heads in $n$ independent tosses of a coin.
\end{example}
\begin{example}[Geometric PMF]
    \begin{align}
        p_{X}(k) &= (1-p)^{k-1} p
    \end{align}
    One example: the number of tosses until the first head appears.
\end{example}
\begin{example}[Poisson PMF]
    \begin{align}
        p_{X}(k) = \mathrm{e}^{-\lambda} \frac{\lambda^{k}}{k!}
    \end{align}
    It's a good approximation of the binomial distribution when $n$ is large and $p$ is small.
    \begin{proof}
        Let $p = \frac{\lambda}{n}$, then
        \begin{align}
            \binom{n}{k} p^{k} (1-p)^{n-k} &= \frac{n!}{k!(n-k)!} p^{k} (1-p)^{n-k} \\
            &= \frac{n(n-1)\cdots(n-k+1)}{k!} p^{k} (1-p)^{n-k} \\
            &\approx \frac{n^{k}}{k!} p^{k} (1-p)^{n-k} \\
            &= \frac{n^{k}}{k!} \left(\frac{\lambda}{n}\right)^{k} \left(1-\frac{\lambda}{n}\right)^{n} \\
            &\approx \frac{n^{k}}{k!} \left(\frac{\lambda}{n}\right)^{k} \mathrm{e}^{-\lambda} \\
            &= \mathrm{e}^{-\lambda} \frac{\lambda^{k}}{k!}
        \end{align}
    \end{proof}
\end{example}
Note that all these PMFs are normalized.

\subsection{Functions of Random Variables}
\begin{definition}[Function of Random Variables]
    Given a RV $X$, a function $Y = g(X)$ is also a RV. If $X$ is discrete, then $Y$ is also discrete, with PMF $p_{Y}$ given by
    \begin{align}
        p_{Y}(y) = \sum_{x: g(x) = y} p_{X}(x)
    \end{align}
\end{definition}
\begin{remark}
    The function $g$ is defined on $\mathbb{R}$, i.e. $g: \mathbb{R} \rightarrow \mathbb{R}$. The meaning of $Y = g(X)$ in fact is $Y(\omega) = g(X(\omega))$.
\end{remark}

\subsection{Expectation and Variance}
\begin{definition}[Expectation of Random Variables]
    The expected value (also called the expectation or the mean) of a random variable $X$, with PMF $p_{X}$, is defined as
    \begin{align}
        \mathbf{E}[X] = \sum_{x} x p_{X}(x)
    \end{align}
\end{definition}
While the PMF gives us the full information about the distribution of $X$, sometimes we still need an overall measure of $X$.
\begin{property}[Properties of Expectation] ~ 
    \begin{itemize}
        \item Let $X$ be a RV and $Y = g(X)$. Sometimes $\mathbf{E}[Y] = \sum_{y} y p_{Y}(y)$ is hard to calculate. Instead, we can use $\mathbf{E}[Y] = \sum_{x} g(x) p_{X}(x)$ which is easier.
        \begin{proof}
            \begin{align}
                \mathbf{E}[Y] &= \sum_{y} y p_{Y}(y) \\
                              &= \sum_{y} y \sum_{x: g(x) = y} p_{X}(x) \\
                              &= \sum_{x} g(x) p_{X}(x)
            \end{align}
        \end{proof}
        \item In general, $\mathbf{E}[g(X)] \neq g(\mathbf{E}[X])$.
        \item If $\alpha_1, \alpha_2, \beta$ are constants, $X, Y$ are RVs, then
        \begin{align}
            \mathbf{E}[\alpha_1 X + \alpha_2 Y + \beta] = \alpha_1 \mathbf{E}[X] + \alpha_2 \mathbf{E}[Y] + \beta
        \end{align}
    \end{itemize}
\end{property}
\begin{definition}[Variance of Random Variables]
    The variance of a RV $X$ is defined as
    \begin{align}
        \mathrm{var}(X) = \mathbf{E}[(X - \mathbf{E}[X])^{2}]
    \end{align}
\end{definition}
Remember $\mathbf{E}[g(x)] = \sum_{x} g(x) p_{X}(x)$, then $\mathrm{var}(X) = \sum_{x} (x - \mathbf{E}[X])^{2} p_{X}(x)$.
We can also define the standard deviation as $\sigma(X) = \sqrt{\mathrm{var}(X)}$.
\begin{property}[Properties of Variance] ~ 
    \begin{itemize}
        \item Define second moment as $\mathbf{E}[X^2] = \sum_x x^2 p_{X}(x)$, then $\mathrm{var}(X)$ can be written in terms of moments expression $\mathrm{var}(X) = \mathbf{E}[X^{2}] - \mathbf{E}[X]^{2}$.
        \item $\mathrm{var}(X) \geq 0$.
        \item $\mathrm{var}(\alpha X + \beta) = \alpha^{2} \mathrm{var}(X)$.
    \end{itemize}
\end{property}
\begin{example}[Expectation and Variance of Binomial Distribution]
    \begin{align}
    &\begin{aligned}
        \mathbf{E}[X] &= \sum_{k=0}^{n} k \binom{n}{k} p^{k} (1-p)^{n-k} \\
                      &= \sum_{k=1}^{n} np \binom{n-1}{k-1} p^{k-1} (1-p)^{n-k} \\
                      &= np \sum_{k=1}^{n} \binom{n-1}{k-1} p^{k-1} (1-p)^{n-k} \\
                      &= np \sum_{k=0}^{n-1} \binom{n-1}{k} p^{k} (1-p)^{n-1-k} \\
                      &= np \cdot \mathrm{PMF} = np
    \end{aligned} \\
    &\begin{aligned}
        \mathbf{E}[X^{2}] =& \sum_{k=0}^{n} k^{2} \binom{n}{k} p^{k} (1-p)^{n-k} \\
                         =& np \sum_{k=1}^{n} k \binom{n-1}{k-1} p^{k-1} (1-p)^{n-k} \\
                         =& np \sum_{k=1}^{n} (k-1+1) \binom{n-1}{k-1} p^{k-1} (1-p)^{n-k} \\ 
                         =& n(n-1)p^2 \sum_{k=2}^{n} \binom{n-2}{k-2} p^{k-2} (1-p)^{n-k} \\ 
                         &+ np \sum_{k=1}^{n} (k-1+1) \binom{n-1}{k-1} p^{k-1} (1-p)^{n-k} \\ 
                         =& n(n-1)p^2 + np
    \end{aligned} \\
    &\begin{aligned}
        \mathrm{var}(X) &= \mathbf{E}[X^{2}] - \mathbf{E}[X]^{2} \\
                        &= n(n-1)p^2 + np - n^2p^2 \\
                        &= np(1-p)
    \end{aligned}
    \end{align}
    Let $n = 1$, then the binomial distribution is the Bernoulli distribution.
\end{example}
\begin{example}[Expectation and Variance of Poisson Distribution]
    \begin{align}
    &\begin{aligned}
        \mathbf{E}[X] &= \sum_{k=0}^{\infty} k \mathrm{e}^{-\lambda} \frac{\lambda^{k}}{k!} \\
                      &= \lambda \mathrm{e}^{-\lambda} \sum_{k=1}^{\infty} \frac{\lambda^{k-1}}{(k-1)!} \\
                      &= \lambda \mathrm{e}^{-\lambda} \mathrm{e}^{\lambda} \\
                      &= \lambda
    \end{aligned} \\
    &\begin{aligned}
        \mathbf{E}[X^{2}] &= \sum_{k=0}^{\infty} k^{2} \mathrm{e}^{-\lambda} \frac{\lambda^{k}}{k!} \\
                         &= \lambda \mathrm{e}^{-\lambda} \sum_{k=1}^{\infty} k \frac{\lambda^{k-1}}{(k-1)!} \\
                         &= \lambda \mathrm{e}^{-\lambda} \sum_{k=1}^{\infty} (k-1+1) \frac{\lambda^{k-1}}{(k-1)!} \\
                         &= \lambda \mathrm{e}^{-\lambda} \left(\lambda \sum_{k=1}^{\infty} \frac{\lambda^{k-1}}{(k-1)!} + \sum_{k=1}^{\infty} \frac{\lambda^{k-1}}{(k-1)!}\right) \\
                         &= \lambda \mathrm{e}^{-\lambda} (\lambda \mathrm{e}^{\lambda} + \mathrm{e}^{\lambda}) \\
                         &= \lambda^{2} + \lambda
    \end{aligned} \\
    &\begin{aligned}
        \mathrm{var}(X) &= \mathbf{E}[X^{2}] - \mathbf{E}[X]^{2} \\
                        &= \lambda^{2} + \lambda - \lambda^{2} \\
                        &= \lambda
    \end{aligned}
    \end{align}
\end{example}

\section{Multiple Random Variables}
\subsection{Conditioning}
\begin{definition}[Conditional PMF]
    The conditional PMF of $X$ given $A$ is defined as
    \begin{align}
        p_{X|A}(x) = \mathbf{P}(X = x | A) = \frac{\mathbf{P}(\{X = x\} \cap A)}{\mathbf{P}(A)}
    \end{align}
    Specifically, if given $Y = y$, then
    \begin{align}
        p_{X|Y = y}(x) = \mathbf{P}(X = x | Y = y) = \frac{\mathbf{P}(\{X = x\} \cap \{Y = y\})}{\mathbf{P}(Y = y)}
    \end{align}
\end{definition}
Since $\mathbf{P}(A) = \sum_{x} \mathbf{P}(\{X = x\} \cap A)$, we have $\sum_{x} p_{X|A}(x) = 1$.
The conditional PMF limits the sample space to $A$.
\begin{definition}[Conditional Expectation]
    The conditional expectation of $X$ given $A$ is defined as
    \begin{align}
        \mathbf{E}[X|A] = \sum_{x} x p_{X|A}(x)
    \end{align}
    Specifically, if given $Y = y$, then
    \begin{align}
        \mathbf{E}[X|Y = y] = \sum_{x} x p_{X|Y}(x|y)
    \end{align}
\end{definition}
\begin{theorem}[Total Expectation Theorem]
    Partition the sample space into disjoint events $A_{1}, A_{2}, \cdots$, then
    \begin{equation}
    \begin{aligned}
        \mathbf{P}(B) &= \sum_{i} \mathbf{P}(A_{i}) \mathbf{P}(B|A_{i}) \\ 
        p_{X}(x) &= \sum_{i} \mathbf{P}(A_{i}) p_{X|A_{i}}(x) \\
        \mathbf{E}[X] &= \sum_{i} \mathbf{P}(A_{i}) \mathbf{E}[X|A_{i}]
    \end{aligned}
    \end{equation}
    From line 2 to line 3, we multiply $x$ on both sides and sum over $x$.
    \begin{figure}[H]
        \centering
        \includegraphics[width=0.3\textwidth]{images/Note-5.1.png}
        \caption{Total Expectation Theorem}
    \end{figure}
\end{theorem}
\begin{example}[Geometric PMF 1]
    Toss a fair coin independently until a head occurs. Let $X$ be the number of tosses. What is the PMF, expectation, and variance of $X$? \\ 
    \textbf{Solution 1:} 
    \begin{equation}
    \begin{aligned}
        \mathbf{E}[X] &= \sum_{k=1}^{\infty} k \mathbf{P}(X = k) = \sum_{k=1}^{\infty} k (1-p)^{k-1} p = \frac{1}{p} \\ 
        \text{var}(X) &= \mathbf{E}[X^{2}] - \mathbf{E}[X]^{2} = \frac{1-p}{p^{2}}
    \end{aligned}
    \end{equation}
\end{example}
\begin{property}[Memoryless Property]
    Given that $X > 2$, the random variable $X - 2$ has same geometric PMF with $X$ (not given that $X > 2$). 
    \begin{equation}
    \begin{aligned}
        p_{X|X > 2}(x) &= \frac{\mathbf{P}(\{X > 2\} \cap \{X = x\})}{\mathbf{P}(X > 2)} \\ 
        &= \frac{(1 - p)^{k-1}p}{1 - p - p(1-p)} = (1-p)^{k-3}p
    \end{aligned}
    \end{equation}
    If we \textit{shift} $k$ by 2, then the PMF is the same. That is to say, the random variable $X - 2$ given $X > 2$ has the same geometric PMF with $X$, and thus $\mathbf{E}[(X - 2)|X > 2] = \mathbf{E}[X]$.
\end{property}
\begin{example}[Geometric PMF 2]
    Toss a fair coin independently until a head occurs. Let $X$ be the number of tosses. What is the PMF, expectation, and variance of $X$? \\ 
    \textbf{Solution 2:} If we use the memoryless property, then 
    \begin{equation}
    \begin{aligned}
        \mathbf{E}[X] &= \mathbf{P}(X = 1)\mathbf{E}[X | X = 1] + \mathbf{P}(X > 1)\mathbf{E}[X | X > 1] \\ 
        &= p + (1-p)\mathbf{E}[(X - 1 + 1) | X > 1] \\
        &= p + (1-p)(1 + \mathbf{E}[X]) \\ 
        &\Rightarrow \mathbf{E}[X] = \frac{1}{p}
    \end{aligned}
    \end{equation}
    Similarly
    \begin{equation}
    \begin{aligned}
        \mathbf{E}[X^2] &= \mathbf{P}(X = 1)\mathbf{E}[X^2 | X = 1] + \mathbf{P}(X > 1)\mathbf{E}[X^2 | X > 1] \\
        &= p + (1-p)\mathbf{E}[(X - 1 + 1)^2 | X > 1] \\
        &= p + (1-p)(1 + 2\mathbf{E}[X] + \mathbf{E}[X^2]) \\
        &\Rightarrow \mathbf{E}[X^2] = \frac{2-p}{p^2}
    \end{aligned}
    \end{equation}
    Then
    \begin{equation}
        \text{var}(X) = \mathbf{E}[X^2] - \mathbf{E}[X]^2 = \frac{1-p}{p^2}
    \end{equation}
\end{example}

\subsection{Multiple Discrete Random Variables}
\begin{definition}[Joint PMF]
    The joint PMF of $X$ and $Y$ is defined as
    \begin{align}
        p_{X, Y}(x, y) = \mathbf{P}(\{X = x\} \cap \{Y = y\})
    \end{align}
\end{definition}
\begin{property}[Properties of Joint PMF] ~
    \begin{itemize}
        \item Added up to 1: $\sum_x \sum_y p_{X, Y}(x, y) = 1$.
        \item Marginal PMF: $p_{X}(x) = \sum_y p_{X, Y}(x, y)$, $p_{Y}(y) = \sum_x p_{X, Y}(x, y)$. (Total Probability Theorem)
        \item Conditional PMF: $p_{X|Y}(x|y) = \frac{p_{X, Y}(x, y)}{p_{Y}(y)}$, $p_{Y|X}(y|x) = \frac{p_{X, Y}(x, y)}{p_{X}(x)}$.
    \end{itemize}
\end{property}
If we have a RV function of multiple RVs: $Z = g(X, Y)$, then the PMF of $Z$ is
\begin{align}
    p_{Z}(z) = \sum_{x, y: g(x, y) = z} p_{X, Y}(x, y)
\end{align}
\begin{definition}[Expectation of Multiple Random Variables]
    Recall that 
    \begin{align*}
        \mathbf{E}[g(X)] = \sum_{x} g(x) p_{X}(x)
    \end{align*}
    Then the expectation of $g(X, Y)$ is
    \begin{align}
        \mathbf{E}[g(X, Y)] = \sum_{x, y} g(x, y) p_{X, Y}(x, y)
    \end{align}
\end{definition}
\begin{property}[Properties of Expectation] ~
    \begin{itemize}
        \item In general, $\mathbf{E}[X + Y] \neq \mathbf{E}[X] + \mathbf{E}[Y]$.
        \item $\mathbf{E}[\alpha X + \beta Y + \gamma] = \alpha \mathbf{E}[X] + \beta \mathbf{E}[Y] + \gamma$. \textbf{This property holds for any RVs, despite of the independence.} (go back to the definition of expectation and you will see the linearity, which is also the reason of property 1.)
    \end{itemize}
\end{property}
\begin{example}[Coin tossing]
    Toss a coin independently. The coin may not be fair. \\ 
    \textbf{Solution:} Define 
    \begin{equation}
        X_i = \begin{cases}
            1, & \text{if the $i$-th toss is head} \\ 
            0, & \text{otherwise}
        \end{cases}
    \end{equation}
    The sum $X = \sum_{i=1}^n X_i$. Then 
    \begin{equation}
        \mathbf{E}[X] = \sum_{i=1}^n \mathbf{E}[X_i] = np
    \end{equation}
    \begin{equation}
    \begin{aligned}
        \text{var}(X) &= \mathbf{E}[X^2] - \mathbf{E}[X]^2 = (\sum_{i=1}^{N} X_i^2 + \sum_{i \neq j} X_i X_j) - n^2 p^2 \\
        &= np + n(n-1)p^2 - n^2 p^2 = np(1-p)
    \end{aligned}
    \end{equation}
\end{example}

\subsection{Independence}
\begin{definition}[Definition of Independence]
    Two RVs $X$ and $Y$ are independent if
    \begin{align}
        p_{X, Y}(x, y) = p_{X}(x) \cdot p_{Y}(y)
    \end{align}
    for all $x$ and $y$.
\end{definition}
\begin{property}[Properties of Independence] ~
    \begin{itemize}
        \item $\mathbf{E}[g(X)h(Y)] = \mathbf{E}[g(X)] \cdot \mathbf{E}[h(Y)]$. (go back to the definition of expectation, divide the sum into two parts, and you will see the independence.)
        \item $\text{var}(g(X) + h(Y)) = \text{var}(g(X)) + \text{var}(h(Y))$. (use $\text{var}(A + B) = \mathbf{E}[(A + B)^2] - \mathbf{E}[A + B]^2$.)
    \end{itemize}
\end{property}

\begin{definition}[Definition of Conditional Independence]
    Two RVs $X$ and $Y$ are conditionally independent given $A$ if
    \begin{align}
        p_{X, Y|A}(x, y) = p_{X|A}(x) p_{Y|A}(y)
    \end{align}
    for all $x$ and $y$.
\end{definition}
\begin{example}[Data packet problem]
    A network system with n nodes randomly redistributes each node's data packet through a central processor. Let $X$ be the number of nodes that receive their original data packets. Find $\mathbf{E}[X]$ and $\text{var}(X)$. \\
    \textbf{Solution:} Define $X_i$ as the indicator of the $i$-th node. Then $X = \sum_{i=1}^n X_i$. 
    \begin{equation}
        \mathbf{E}[X] = \sum_{i=1}^{n} \mathbf{E}[X_i] = n \cdot \frac{1}{n} = 1
    \end{equation}
    \begin{equation}
    \begin{aligned}
        \text{var}(X) &= \mathbf{E}[X^2] - \mathbf{E}[X]^2 = \sum_{i=1}^{n} \mathbf{E}[X_i^2] + \sum_{i \neq j} \mathbf{E}[X_i X_j] - 1 \\ 
        &= n \cdot \frac{1}{n} + \sum_{i=1}^{n} \left(\mathbf{P}(X_i = 1) \cdot \mathbf{P}(X_j = 1 | X_i = 1)\right) - 1 \\ 
        &= n(n-1) \cdot \frac{1}{n(n-1)} = 1
    \end{aligned}
    \end{equation}
\end{example}



\section{Continuous Random Variables}
\subsection{Continuous RVs and PDFs}
\begin{definition}[Continuous Random Variables]
    A RV $X$ is continuous if there exists a nonnegative function $f_{X}(x)$, called \textbf{probability density function}, such that
    \begin{align}
        \mathbf{P}(X \in A) = \int_{A} f_{X}(x) \dd{x}
    \end{align}
    Especially, if $A \subseteq \mathcal{B}(\mathbb{R})$, then
    \begin{align}
        \mathbf{P}(a \leq X \leq b) = \int_{a}^{b} f_{X}(x) \dd{x}
    \end{align}
\end{definition}
\begin{property}[Properties of PDF]
    Let $X$ be a continuous RV with PDF $f_{X}(x)$, then
    \begin{itemize}
        \item $f_{X}(x) \geq 0$, and $f_{X}(x)$ can be any nonnegative number. (even $+\infty$)
        \item $\int_{-\infty}^{\infty} f_{X}(x) \dd{x} = 1$.
        \item $\mathbf{P}(x \leq X \leq x + \delta) = \int_{x}^{x + \delta} f_{X}(x) \dd{x} \approx f_{X}(x) \cdot \delta$.
    \end{itemize}
\end{property}

\subsection{Expectation and Variance}
\begin{definition}[Definitions of Expectation and Variance]
    Let $X$ be a continuous RV with PDF $f_{X}(x)$, then
    \begin{equation}
    \begin{aligned}
        &\mathbf{E}[X] = \int_{-\infty}^{\infty} x f_{X}(x) \dd{x} \\ 
        &\mathbf{E}[g(X)] = \int_{-\infty}^{\infty} g(x) f_{X}(x) \dd{x}
    \end{aligned}
    \end{equation}
    \begin{equation}
    \begin{aligned}
        \text{var}(X) &= \mathbf{E}[(X - \mathbf{E}[X])^{2}] = \int_{-\infty}^{\infty} (x - \mathbf{E}[X])^{2} f_{X}(x) \dd{x} \\ 
        &= \mathbf{E}[X^{2}] - \mathbf{E}[X]^{2}
    \end{aligned}
    \end{equation}
\end{definition}
\begin{example}[Uniform RV]
    Consider a RV $X$ that takes value in an interval $[a, b]$ with PDF
    \begin{figure}[H]
        \centering
        \includegraphics[width=0.3\textwidth]{images/Note-6.1.png}
    \end{figure}
    \vspace{-2em}
    \begin{equation}
        \mathbf{E}[X] = \int_{a}^{b} \frac{x}{b - a} \dd{x} = \frac{a + b}{2}
    \end{equation}
    \begin{equation}
        \text{var}(X) = \mathbf{E}[X^2] - (\mathbf{E}[X])^2 = \int_{a}^{b} \frac{x^2}{b - a} \dd{x} - \left(\frac{a + b}{2}\right)^2 = \frac{(b - a)^2}{12}
    \end{equation}
\end{example}
\begin{example}[Exponential RV]
    Consider an exponential RV $X$ that has a PDF of the form
    \begin{equation}
        f_{X}(x) = \lambda \mathrm{e}^{-\lambda x},~ x \geq 0
    \end{equation}
    \begin{equation}
        \mathbf{E}[X] = \int_{0}^{\infty} x \lambda \mathrm{e}^{-\lambda x} \dd{x} = (-x \mathrm{e}^{-\lambda x}) \Big|_{0}^{\infty} + \int_{0}^{\infty} \mathrm{e}^{-\lambda x} \dd{x} = \frac{1}{\lambda}
    \end{equation}
    \begin{equation}
        \text{var}(X) = \mathbf{E}[X^2] - (\mathbf{E}[X])^2 = (-x^2 \mathrm{e}^{-\lambda x}) \Big|_{0}^{\infty} + \int_{0}^{\infty} 2x \mathrm{e}^{-\lambda x} \dd{x} - \frac{1}{\lambda^2} = \frac{1}{\lambda^2}
    \end{equation}
\end{example}

\subsection{Comulative Distribution Function}
We want to describe both discrete and continuous RVs in a unified way. An intuition is to ``accumulate'' the probability ``up to'' the value $x$.
\begin{definition}[CDF]
    The cumulative distribution function (CDF) of a RV $X$ is defined as
    \begin{align}
        F_{X}(x) = \mathbf{P}(X \leq x) = \left\{
        \begin{aligned}
            &\sum_{t \leq x} p_{X}(t), &\text{discrete} \\ 
            &\int_{-\infty}^{x} f_{X}(t) \dd{t}, &\text{continuous}
        \end{aligned}
        \right.
    \end{align}
\end{definition}
\begin{example}[Some examples of CDF] ~
    \begin{figure}[H]
        \centering
        \includegraphics[width=0.6\textwidth]{images/Note-6.2.png}
        \caption{CDF: Continuous RV}
    \end{figure}
    \begin{figure}[H]
        \centering
        \includegraphics[width=0.6\textwidth]{images/Note-6.3.png}
        \caption{CDF: Discrete RV}
    \end{figure}
\end{example}
\begin{property}[Properties of CDF] ~
    \begin{itemize}
        \item Monotonically nondecreasing: $x \leq y \Rightarrow F_{X}(x) \leq F_{X}(y)$. (since $p_{X}(x) \geq 0$ and $f_{X}(x) \geq 0$)
        \item $F_X(x) \to 0$ as $x \to -\infty$ and $F_X(x) \to 1$ as $x \to \infty$.
        \item When $X$ is continuous: $F_{X}(x)$ is a continuous function, $f_{X}(x) = \frac{\dv{F_{X}}}{\dd{x}}(x)$.
        \item When $X$ is discrete: $F_{X}(x)$ i piecewise constant function, $p_{X}(k) = F_{X}(k) - F_{X}(k-1)$.
    \end{itemize}
\end{property}

\end{document}
