\chapter{Probability Space}
\begin{definition}[Probability Space]
    Probability space is a tuple $\left(\varOmega,~\mathcal{F},~\mathbf{P}\right)$ where
    \begin{itemize}
        \item Sample space $\varOmega$ is a set of all possible outcomes.
        \item $ \sigma $-algebra $\mathcal{F}$ is a collection of subsets of $\varOmega$.
        \item Probability measure $\mathbf{P}$ is a function that assigns a probability to each event in $\mathcal{F}$.
    \end{itemize}
\end{definition}

\begin{definition}[Sample Space]
    Sample space $\varOmega$ is a set of all possible outcomes, which should satisfy the following properties:
    \begin{itemize}
        \item Mutually Exclusive: The elements in $\varOmega$ should be unique.
        \item Collectively Exhaustive: The elements in $\varOmega$ should cover all possible outcomes.
    \end{itemize}
\end{definition}

Once the experiment is conducted, there is \textbf{exactly one} element in the sample space $\varOmega$ occurs.

\begin{example}[Example of Sample Space]
    Here are some examples of sample space:
    \begin{itemize}
        \item Discrete: $\varOmega = \{1, 2, 3, 4, 5, 6\}$ - Rolling a fair die.
        \item Continuous: $\varOmega = [0, 1]$ - Randomly selecting a real number between 0 and 1.
    \end{itemize}
\end{example}

Before we define the $\sigma$-algebra, we need to introduce the concept of \textbf{countable}. We say a set $X$ is countable if there is a bijection between $X$ and $\mathbb{N}$. Some countable sets are: $\mathbb{N}$, $\mathbb{Z}$, $\mathbb{Q}$, $\mathbb{N}^2$, $\mathbb{Z}^2$, $\mathbb{Q}^2$, $\cup_{n=1}X_{n}$. Some uncountable sets are: $\mathbb{R}$, $\mathbb{C}$, $\mathbb{R}^2$, $\mathbb{C}^2$.

\begin{definition}[$\sigma$-algebra]
    We say a set $\mathcal{F}$  collection of subsets of $\varOmega$ is a $\sigma$-algebra if it satisfies the following properties:
\end{definition}

\begin{remark}
    Not all the things in real worlf can be measured by probability, especially when things are continuous (uncountable).
\end{remark}
