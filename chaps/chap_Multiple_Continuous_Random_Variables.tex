\chapter{Multiple Continuous Random Variables}


\section{Multiple Continuous Random Variables}

\begin{definition}[Joint PDF]
    The two continuous RVs $X$ and $Y$, with the same experiment, are jointly continuous if they can be described by a joint PDF $f_{X, Y}(x, y)$, where $f_{X, Y}(x, y) \geq 0$ and satisfies
    \begin{equation}
        \mathbf{P}((X, Y) \in A) = \iint_{A} f_{X, Y}(x, y) \dd{x} \dd{y}
    \end{equation}
    Especially, if $A \subseteq \mathcal{B}(\mathbb{R}^2)$, then
    \begin{equation}
        \mathbf{P}((X, Y) \in [a, b] \times [c, d]) = \int_{c}^{d} \int_{a}^{b} f_{X, Y}(x, y) \dd{x} \dd{y}
    \end{equation}
\end{definition}

\begin{property}[Properties of Joint PDFs] ~
    \begin{itemize}
        \item Normalization: $\iint_{\Omega} f_{X, Y}(x, y) \dd{x} \dd{y} = 1$.
        \item Interpretation: $f_{X, Y}(x, y) \dd{x} \dd{y}$ is the probability that $(X, Y)$ lies in the small rectangle $[x, x + \dd{x}] \times [y, y + \dd{y}]$.
        \item Marginal PDFs: $f_{X}(x) = \int_{\Omega_{Y}} f_{X, Y}(x, y) \dd{y}$ and $f_{Y}(y) = \int_{\Omega_{X}} f_{X, Y}(x, y) \dd{x}$.
    \end{itemize}
\end{property}

\begin{definition}[Joint CDFs]
    If $X$ and $Y$ are two continuous RVs associated with the same experiment, then the joint CDF of $X$ and $Y$ is defined as
    \begin{equation}
        F_{X, Y}(x, y) = \mathbf{P}(X \leq x, Y \leq y)
    \end{equation}
    If $X$ and $Y$ can be described by a joint PDF $f_{X, Y}(x, y)$, then
    \begin{equation}
        F_{X, Y}(x, y) = \int_{-\infty}^{x} \int_{-\infty}^{y} f_{X, Y}(t, s) \dd{t} \dd{s}
    \end{equation}
    Conversely, if $F_{X, Y}(x, y)$ is differentiable, then
    \begin{equation}
        f_{X, Y}(x, y) = \pdv{F_{X, Y}(x, y)}{x}{y}
    \end{equation}
\end{definition}

\begin{definition}[Expectation and Variance]
    Let $X$ and $Y$ be two continuous RVs with joint PDF $f_{X, Y}(x, y)$, then
    \begin{equation}
    \begin{aligned}
        &\mathbf{E}[g(X, Y)] = \iint_{\Omega} g(x, y) f_{X, Y}(x, y) \dd{x} \dd{y} \\ 
        &\text{var}(g(X, Y)) = \mathbf{E}[g(X, Y) - \mathbf{E}[g(X, Y)]]^2 = \mathbf{E}[g(X, Y)^2] - \mathbf{E}[g(X, Y)]^2
    \end{aligned}
    \end{equation}
\end{definition}

\begin{definition}[Independence]
    $X$ and $Y$ are independent if their joint PDF can be factorized as
    \begin{equation}
        f_{X, Y}(x, y) = f_{X}(x) f_{Y}(y), ~\forall x, y
    \end{equation}
\end{definition}

\begin{example}[Buffon's needle]
    Parallel lines are at distance $d$, throw a needle of length $l$ (assume $l < d$). Find $\mathbf{P}(\text{needle intersects one of the lines})$.  
    \begin{solution}
        Let $X$ be the distance from the center of the needle to the nearest line, and $\Theta$ be the angle between the needle and the lines, where $X$ and $\Theta$ are independent. Given the symmetry characteristics, it is well to assume $X$ is uniform in $[0, d/2]$ and $\Theta$ is uniform in $[0, \pi/2]$. Then the joint PDF is
    \begin{equation}
        f_{X, \Theta}(x, \theta) = \frac{2}{d} \cdot \frac{2}{\pi}, \quad 0 \leq x \leq \frac{d}{2}, 0 \leq \theta \leq \frac{\pi}{2}
    \end{equation}
    The needle intersects if $X \leq l/2 \sin\Theta$, and the probability is
    \begin{equation}
    \begin{aligned}
        \mathbf{P}(X \leq \frac{l}{2}\sin\Theta) &= \iint_{X\leq\frac{l}{2}\sin\Theta} f_{X, \Theta}(x, \theta) \dd{x} \dd{\theta} \\ 
        &= \frac{4}{\pi d} \int_{0}^{\pi/2} \int_{0}^{(l/2)\sin\theta} \dd{x} \dd{\theta} = \frac{2l}{\pi d}
    \end{aligned}
    \end{equation}
    \end{solution}
\end{example}


\section{Conditioning and Independence}

\begin{definition}[Definition of Conditional PDFs]
    Let $X$ and $Y$ be two continuous RVs with joint PDF $f_{X, Y}(x, y)$, then the conditional PDF of $X$ given $Y = y$ is defined as
    \begin{equation}
        f_{X \mid Y}(x \mid y) = \frac{f_{X, Y}(x, y)}{f_{Y}(y)}
    \end{equation}
\end{definition}

\begin{remark}
    There's a paradox that the denominator $\mathbf{P}(Y = y) = 0$ for continuous RVs, but we can let $Y$ be a little fatter: $\mathbf{P}(y \leq Y \leq y + \delta) \approx f_{Y}(y) \cdot \delta$ and let $\delta \to 0$ after we do division.
\end{remark}

\begin{figure}[H]
    \centering
    \includegraphics[width=0.5\textwidth]{images/Note-7.1.png}
    \caption{Visualization of Conditional PDF}
\end{figure}

\begin{example}[Stick-breaking]
    \label{ex:stick-breaking}
    Break a stick of length $l$ twice: break at $X$: uniform in $[0, l]$; break again at $Y$: uniform in $[0, X]$. Find $\mathbf{E}[Y]$. 
    \begin{solution}
        \begin{align}
            f_{X, Y}(x, y) &= f_{X}(x) f_{Y \mid X}(y \mid x) = \frac{1}{l} \cdot \frac{1}{x}, \quad 0 \leq y \leq x \leq l \\ 
            f_{Y}(y) &= \int_{-\infty}^{+\infty} f_{X, Y}(x, y) \dd{x} = \int_{y}^{l} \frac{1}{l} \cdot \frac{1}{x} \dd{x} = \frac{1}{l} \ln\left(\frac{l}{y}\right), \quad 0 \leq y \leq l \\ 
            \mathbf{E}[Y] &= \int_{-\infty}^{+\infty} y f_{Y}(y) \dd{y} = \int_{0}^{l} y \cdot \frac{1}{l} \ln\left(\frac{l}{y}\right) \dd{y} = \frac{l}{4}
        \end{align}
    \end{solution}
\end{example}

\begin{definition}[Conditional Expectation]
    The conditional expectation of $X$ given that $Y = y$ has happened is defined by
    \begin{equation}
        \mathbf{E}[X \mid Y = y] = \int_{-\infty}^{+\infty} x f_{X \mid Y}(x \mid y) \dd{x}
    \end{equation}
    For a function $g(X, Y)$, the conditional expectation is 
    \begin{equation}
        \mathbf{E}[g(X, Y) \mid Y = y] = \int_{-\infty}^{+\infty} g(x, y) f_{X \mid Y}(x \mid y) \dd{x}
    \end{equation}
\end{definition}

Remember how we prove the total expectation theorem in discrete case: we multiple $x$ on both side of the equation of conditional probability and sum over $x$. Now we can do the same thing in continuous case.
\begin{theorem}[Total Expectation Theorem]
    Let $X$ and $Y$ be two continuous RVs, then
    \begin{equation}
        \mathbf{E}[X] = \mathbf{E}_{Y}\left[\mathbf{E}_{X \mid Y}[X \mid Y]\right] =  \int_{-\infty}^{+\infty} \mathbf{E}[X \mid Y = y] f_{Y}(y) \dd{y}
    \end{equation}
\end{theorem}

\begin{solution}
    \textbf{Another way to solve Example \ref{ex:stick-breaking}.} \\ 
    Conditioning on $X = x$, we have
    \begin{equation}
        \mathbf{E}[Y \mid X = x] = \int_{0}^{x} y f_{Y \mid X}(y \mid x) \dd{y} = \int_{0}^{x} y \cdot \frac{1}{x} \dd{y} = \frac{x}{2}
    \end{equation}
    With total expectation theorem, we have
    \begin{equation}
        \mathbf{E}[Y] = \mathbf{E}_{X}\left[\mathbf{E}[Y \mid X]\right] = \int_{0}^{l} \frac{x}{2} f_{X}(x) \dd{x} = \int_{0}^{l} \frac{x}{2l} \dd{x} = \frac{l}{4}
    \end{equation}
\end{solution}

\begin{definition}[Independence]
    Two continuous RVs $X$ and $Y$ are independent if their joint PDF can be factorized as
    \begin{equation}
        f_{X, Y}(x, y) = f_{X}(x) f_{Y}(y), ~\forall x, y
    \end{equation}
\end{definition}
Similar to discrete case, the following properties hold:
\begin{property}[Properties of Independence]
    \begin{itemize}
        \item $\mathbf{E}[g(X)h(Y)] = \mathbf{E}[g(X)]\mathbf{E}[h(Y)]$ if $X$ and $Y$ are independent.
        \item $\text{var}(X + Y) = \text{var}(X) + \text{var}(Y)$ if $X$ and $Y$ are independent.
    \end{itemize}
\end{property}


\section{Bayes' Theorem}
Change a little bit from $f_{X, Y}(x, y) = f_{X \mid Y}(x \mid y) f_{Y}(y)$, we have
\begin{theorem}[Bayes' Theorem]
    Let $X$ and $Y$ be two continuous RVs, then
    \begin{equation}
        f_{X \mid Y}(x \mid y) = \frac{f_{Y \mid X}(y \mid x) f_{X}(x)}{f_{Y}(y)} = \frac{f_{Y \mid X}(y \mid x) f_{X}(x)}{\int_{-\infty}^{+\infty} f_{Y \mid X}(y \mid t) f_{X}(t) \dd{t}}
    \end{equation}
    which can be interpreted using $f(x)\dd{x} = \mathbf{P}(x \leq X \leq x + \dd{x})$.
\end{theorem}
