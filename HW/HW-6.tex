\documentclass[utf8]{article}
% \usepackage{ctex}
% \usepackage{circuitikz}
% \usepackage{tikz}
% \usepackage{graphicx}
\usepackage{hyperref}
\usepackage{geometry}
\usepackage{pdfpages}
\usepackage{booktabs}
\usepackage{subfigure}
\usepackage{float}
\usepackage{amsmath}
\usepackage{amssymb}
\usepackage{amsthm}
% math font
\usepackage{mathrsfs}
\usepackage{multirow}
\usepackage{inputenc}
\usepackage{fancyhdr}
\usepackage{enumitem}
\usepackage{physics}

\newcommand{\e}{\mathrm{e}}


\theoremstyle{definition}% default
\newtheorem{question}{Question} %
\theoremstyle{plain}% 
\newtheorem{answer}{Answer} %


\title{Problem Set 6}
\author{ FHYQ-Dong, Class **********, Student ID: ********** }
\date{\today}

\geometry{a4paper, scale=0.8}
\setlength{\columnsep}{20pt}
\pagestyle{fancy}
\fancyhf{}
\renewcommand{\headrulewidth}{0.5pt}
\renewcommand{\footrulewidth}{0.5pt}
\lhead{Problem Set 6}
\rhead{ FHYQ-Dong }
\cfoot{---~\thepage~---}

\hypersetup{
    colorlinks=true,
    linkcolor=black,
    filecolor=black,      
    urlcolor=black,
    citecolor=black,
}


\begin{document}

\maketitle
\thispagestyle{fancy}

\begin{question}[Laplace random variable]
    Let X have the PDF
    \begin{align}
        f_{X}(x) = \frac{\lambda}{2}\e^{-\lambda|x|}
    \end{align}
    where $\lambda$ is a positive scalar. Verify that $f_X$ satisfies the normalization condition, and evaluate the mean and variance of $X$.
\end{question}
\begin{answer}
    \begin{align}
    &\begin{aligned}
        \int_{-\infty}^{+\infty}f_{X}(x)dx &= \int_{-\infty}^{+\infty}\frac{\lambda}{2}\e^{-\lambda|x|}dx \\
        &= \int_{-\infty}^{0}\frac{\lambda}{2}\e^{\lambda x}dx + \int_{0}^{+\infty}\frac{\lambda}{2}\e^{-\lambda x}dx \\
        &= \frac{\lambda}{2}\left(\frac{1}{\lambda} - 0 - 0 + \frac{1}{\lambda}\right) \\
        &= 1
    \end{aligned} \\ 
    &\begin{aligned}
        \mathbf{E}[X] &= \frac{\lambda}{2}\int_{-\infty}^{0}xe^{\lambda x}dx + \frac{\lambda}{2}\int_{0}^{+\infty}xe^{-\lambda x}dx \\
        &= \frac{\lambda}{2}\cdot\frac{1}{\lambda}\left(\left(xe^{\lambda x}\Big|_{-\infty}^{0} - \int_{-\infty}^{0}\e^{\lambda x}\dd{x}\right) - \left(xe^{-\lambda x}\Big|_{0}^{+\infty} - \int_{0}^{+\infty}\e^{-\lambda x}\dd{x}\right)\right) \\ 
        &= \frac{1}{2} \cdot \frac{-1 + 0 - 0 + 1}{\lambda} \\
        &= 0
    \end{aligned} \\ 
    &\begin{aligned}
        \text{var}(X) &= \mathbf{E}[X^2] - (\mathbf{E}[X])^2 \\
        &= \int_{-\infty}^{+\infty}x^2f_{X}(x)dx - 0 \\ 
        &= \left(\int_{-\infty}^{0}x^2\frac{\lambda}{2}\e^{\lambda x}dx + \int_{0}^{+\infty}x^2\frac{\lambda}{2}\e^{-\lambda x}dx\right) \\
        &= \frac{\lambda}{2}\cdot\frac{1}{\lambda} \left(\left(x^2e^{\lambda x}\Big|_{-\infty}^{0} - \int_{-\infty}^{0}2xe^{\lambda x}\dd{x}\right) - \left(x^2e^{\lambda x} - \int_{0}^{+\infty}2xe^{-\lambda x}\dd{x}\right)\right) \\
        &= \frac{1}{2} \cdot \frac{1}{\lambda} \left(\left(-2xe^{\lambda x}\Big|_{-\infty}^{0} + 2\int_{-\infty}^{0}\e^{\lambda x}\dd{x}\right) - \left(2xe^{-\lambda x}\Big|_{0}^{+\infty} - 2\int_{0}^{+\infty}\e^{-\lambda x}\dd{x}\right)\right) \\
        &= \frac{1}{\lambda^2} \left(1 - 0 - 0 + 1\right) = \frac{2}{\lambda^2}
    \end{aligned}
    \end{align}
\end{answer}

\begin{question}
    Alvin throws darts at a circular target of radius $r$ and is equally likely to hit any point in the target. Let $X$ be the distance of Alvin's hit from the center.
    \begin{enumerate}[label=(\alph*)]
        \item Find the PDF, the mean and the variance of $X$.
        \item The target has an inner circle of radius $t$. If $X < t$, Alvin gets a score of $S = 1/X$. Otherwise his score is $S = 0$. Find the CDF of $S$. Is $S$ a continuous random variable?
    \end{enumerate}
\end{question}
\begin{answer} ~
    \begin{enumerate}[label=(\alph*)]
        \item The probability of hitting a circle with radius $x$ is $\frac{x^2}{r^2}$, which is the CDF of $X$. So the PDF of $X$ is
        \begin{equation}
            f_{X}(x) = \dv{F_{X}(x)}{x} = \dv{\frac{x^2}{r^2}}{x} = \frac{2x}{r^2}, \quad 0 \leq x \leq r
        \end{equation}
        And the mean and variance of $X$ are
        \begin{align}
            \mathbf{E}[X] &= \int_{0}^{r} x\frac{2x}{r^2} \dd{x} = \frac{2r}{3} \\ 
            \text{var}[X] &= \mathbf{E}[X^2] - (\mathbf{E}[X])^2 = \int_{0}^{r} x^2\frac{2x}{r^2} \dd{x} - \left(\frac{2r}{3}\right)^2 = \frac{r^2}{18}
        \end{align}
        \item The probability of getting 0 score is $1 - \frac{t^2}{r^2}$, and the probability of getting non-zero score is $\frac{t^2}{r^2}$. Given $X < t$, the probability of getting a score no more than $\frac{1}{X}$ is $1 - \frac{1}{t^2s^2}$. Therefore, the CDF of $S$ is
        \begin{equation}
            F_{S}(s) = \left\{\begin{aligned}
                &0, \quad &s < 0 \\
                &1 - \frac{t^2}{r^2}, \quad &0 \leq s \leq 1/t \\ 
                &1 - \frac{1}{s^2r^2}, \quad &s > 1/t
            \end{aligned}\right.
        \end{equation}
        Because the CDF of $S$ is piecewise constant, when $s < \frac{1}{t}$, $S$ is not a continuous random variable.
    \end{enumerate}
\end{answer}

\begin{question} ~
    \begin{enumerate}[label=(\alph*)]
        \item A fire station is to be located along a road of length $A$ ($A < \infty$). If fires occur at points uniformly distributed on $(0, A)$, where should the fire station be located so as to minimize the expected distance from the fire?
        \item Now suppose that the road is of infinite length, stretching from point 0 outward to $\infty$. If the distance of a fire from point 0 is exponentially distributed with rate $\lambda$, where should the fire station be located now?
    \end{enumerate}
\end{question}
\begin{answer} ~
    \begin{enumerate}[label=(\alph*)]
        \item Assume the fire station is located at $x_0$ ($x_0 \in (0, A)$), then the PDF of the distance from the fire station to the fire is
        \begin{equation}
            f_{X}(x) = \left\{
            \begin{aligned}
                &2/A, \quad &0 < x < \min(x_0, A - x_0) \\
                &1/A, \quad &\min(x_0, A - x_0) < x < \max(x_0, A - x_0) \\
            \end{aligned}\right.
        \end{equation}
        Due to the symmetry of the problem, it would be well to assume $x_0 \leq A/2$. Then the expected distance from the fire is
        \begin{equation}
            \mathbf{E}[X] = \int_{0}^{x_0}\frac{2x}{A}\dd{x} + \int_{x_0}^{A - x_0}\frac{x}{A}\dd{x} = \frac{x_0^2}{A} + \frac{(A - x_0)^2}{2A} - \frac{x_0^2}{2A} = \frac{(A - x_0)^2 + x_0^2}{2A}
        \end{equation}
        Since $x_0 + (A - x_0) = A = \text{const.}$, the expected distance is minimized when $x_0 = A/2$, i.e., the fire station should be located at the midpoint of the road.
        \item Similarly, we assume the fire station is located at $x_0$ ($x_0 \in (0, \infty)$), then the PDF of the distance from the fire station to the fire is
        \begin{equation}
            f_{X}(x) = \left\{
            \begin{aligned}
                &\lambda \e^{-\lambda x_0}, \quad &x = 0 \\ 
                &\lambda \e^{-\lambda (x_0 - x)} + \lambda \e^{-\lambda (x_0 + x)}, \quad &0 < x < x_0 \\
                &\lambda \e^{-\lambda (x_0 + x)}, \quad &x \geq x_0
            \end{aligned}\right.
        \end{equation}
        Therefore, the expected distance from the fire is
        \begin{equation}
        \begin{aligned}
            \mathbf{E}[X] &= \int_{0}^{x_0}x\lambda \e^{-\lambda (x_0 - x)}\dd{x} + \int_{0}^{+\infty}x\lambda \e^{-\lambda (x_0 + x)}\dd{x} \\ 
            &= \lambda \cdot \frac{1}{\lambda} \left(x_0 - 0 + \frac{1 - \e^{-\lambda x_0}}{\lambda}\right) - \lambda \cdot \frac{1}{\lambda} \left(0 - 0 + \frac{0 - \e^{-\lambda x_0}}{\lambda}\right) \\ 
            &= x_0 + \frac{1}{\lambda}
        \end{aligned}
        \end{equation}
        So the fire station should be located at the origin.
    \end{enumerate}
\end{answer}

\begin{question}
    Show that $\Gamma\left(\frac{1}{2}\right) = \sqrt{\pi}$ \\ 
    \textit{Hint:} $\Gamma(t) = \int_{0}^{+\infty}\e^{-x}x^{t-1}\dd{x}$. Make the change of variables and then relate the resulting expression to the normal distribution.
\end{question}
\begin{answer} ~
    \begin{equation}
    \begin{aligned}
        \Gamma\left(\frac{1}{2}\right) &= \int_{0}^{+\infty} \frac{\e^{-x}}{\sqrt{x}}\dd{x} = \int_{0}^{+\infty} \frac{\e^{-u^2}}{u} \cdot 2u\dd{u} = \int_{-\infty}^{+\infty}\e^{-u^2}\dd{u} = \sqrt{\pi}
    \end{aligned}
    \end{equation}
\end{answer}

\begin{question}
    Suppose $X$ is a random variable with CDF
    \begin{equation}
        F_{X}(x) = \frac{1}{1 + \exp(-\lfloor x \rfloor)}
    \end{equation}
    where $\lfloor x \rfloor$ is the greatest integer less than or equal to $x$. Find $\mathbf{P}(|X| \leq 3)$.
\end{question}
\begin{answer}
    The CDF of $X$ is piecewise constant, so $X$ is a discrete RV, and 
    \begin{equation}
    \begin{aligned}
        \mathbf{P}(|X| \leq 3) &= \sum_{x=-3}^{3}p_{X}(x) = F_{X}(3) - F_{X}(-4) = \frac{1}{1 + \e^{-3}} - \frac{1}{1 + \e^{4}}
    \end{aligned}
    \end{equation}
\end{answer}

\end{document}
