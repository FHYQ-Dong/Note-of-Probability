\documentclass[utf8]{article}
% \usepackage{ctex}
% \usepackage{circuitikz}
% \usepackage{tikz}
% \usepackage{graphicx}
\usepackage{hyperref}
\usepackage{geometry}
\usepackage{pdfpages}
\usepackage{booktabs}
\usepackage{subfigure}
\usepackage{float}
\usepackage{amsmath}
\usepackage{amssymb}
\usepackage{amsthm}
% math font
\usepackage{mathrsfs}
\usepackage{multirow}
\usepackage{multicol}
\usepackage{inputenc}
\usepackage{fancyhdr}
\usepackage{enumitem}
\usepackage{physics}
\usepackage{mathtools}

\theoremstyle{definition}% default
\newtheorem{question}{Question} %
\theoremstyle{plain}% 
\newtheorem{answer}{Answer} %

\newcommand{\e}{\mathrm{e}}


\title{Problem Set 10}
\author{ FHYQ-Dong, Class **********, Student ID: ********** }
\date{\today}

\geometry{a4paper, scale=0.8}
\setlength{\columnsep}{20pt}
\pagestyle{fancy}
\fancyhf{}
\renewcommand{\headrulewidth}{0.5pt}
\renewcommand{\footrulewidth}{0.5pt}
\lhead{Problem Set 9}
\rhead{ FHYQ-Dong }
\cfoot{---~\thepage~---}

\hypersetup{
    colorlinks=true,
    linkcolor=black,
    filecolor=black,      
    urlcolor=black,
    citecolor=black,
}


\begin{document}

\maketitle
\thispagestyle{fancy}

\begin{question}
A pizza parlor serves $n$ different types of pizza, and is visited by a number $K$ of customers in a given period of time, where $K$ is a nonnegative integer random variable with a known associated transform $M_K(s) = \E[\e^{sK}]$. Each customer orders a single pizza. with all types of pizza being equally likely, independent of the number of other customers and the types of pizza they order. Give a formula, in terms $M_K(\cdot)$, for the expected number of different types of pizzas ordered.
\end{question}
\begin{answer}
    Let $X$ be the number of different types of pizzas ordered and $X_i$ be the indicator for the $i$-th type of pizza being ordered by at least one customer, then
    \begin{equation}
        X = \sum_{i=1}^n X_i \quad \text{and} \quad \mathbf{P}(X_i = 1) = 1 - \qty(\frac{n - 1}{n})^K
    \end{equation}
    Given $K$, the expectated number of different types of pizzas ordered is
    \begin{equation}
        \E[N |K] = \sum_{i=1}^n \E[X_i | K] = n \cdot \E[X_1 | K] = n - n \cdot \qty(\frac{n - 1}{n})^K
    \end{equation}
    According to the law of total expectation, we have
    \begin{equation}
        \E[N] = \E[\E[N | K]] = n - n \cdot \E\qty[\qty(\frac{n - 1}{n})^K] = n - n \cdot \E\qty[\e^{K \ln(\frac{n - 1}{n})}] = n - n \cdot M_K\qty(\ln(\frac{n - 1}{n}))
    \end{equation}
\end{answer}

\begin{question}
At a certain time, the number of people that enter an elevator is a Poisson random variable with parameter $\lambda$. The weight of each person is independent of every other persons weight, and is uniformly distributed between 100 and 200 lbs. Let $X_i$ be the fraction of 100 by which the $i$-th person exceeds 100 lbs, e.g., if the 7-th person weighs 175 lbs., then $X_7 = 0.75$. Let $Y$ be the sum of the $X_i$.
\begin{enumerate}[label=(\alph*)]
    \item Find the transform associated with $Y$.
    \item \label{q:2.b} Use the transform to compute the expected value of $Y$. 
    \item Verify your answer to part \ref{q:2.b} by using the law of iterated expectations.
\end{enumerate}
\end{question}
\begin{answer} ~
    \begin{enumerate}[label=(\alph*)]
        \item Let $N \sim \text{Poisson}(\lambda)$ be the number of people that enter the elevator. The transform associated with $Y$ is
        \begin{equation}
            \E[\e^{sY}] = \E[\E[\e^{sY} | N]] = \E\qty[\E\qty[\exp\qty(\sum_{n=1}^{N}sX_n)]] = \E\qty[M_{X}(s)^N] = M_N(\ln(M_X(s))) 
        \end{equation}
        given that $M_N(s) = \e^{\lambda(\e^s - 1)}$ and $M_X(s) = \frac{\e^s - 1}{s}$, $M_Y(s)$ can be expressed as
        \begin{equation}
            M_Y(s) = \exp\qty(\lambda\qty(\e^{\ln(M_X(s))} - 1)) = \exp\qty(\lambda\qty(\frac{\e^s - 1}{s} - 1))
        \end{equation}
        \item The expected value of $Y$ can be computed as follows
        \begin{equation}
            \E[Y] = \dv{s} M_Y(s) \Big|_{s=0} = \lambda \exp\qty(M_X(0) - 1) \cdot \dv{s} M_X(s) \Big|_{s=0} = \frac{\lambda}{2}
        \end{equation}
        \item According to the law of iterated expectations
        \begin{equation}
            \E[Y] = \E[\E[Y | N]] = \E[N \cdot \E[X]] = \E[N] \cdot \E[X] = \lambda \cdot \frac{1}{2} = \frac{\lambda}{2}
        \end{equation}
    \end{enumerate}
\end{answer}

\begin{question}
    Consider the calculation of the mean and variance of a sum
    \begin{equation}
        Y = X_1 + X_2 + \cdots + X_N
    \end{equation}
    where $N$ is itself a sum of integer-valued random variables, i.e.
    \begin{equation}
        N = K_1 + K_2 + \cdots + K_M
    \end{equation}
    Here $N, M, K_1, K_2, \cdots, X_1, X_2, \cdots$ are independent random variables, $N, M, K_1, K_2, \cdots$ are integer valued and nonnegative, $K_1, K_2, \cdots$ are identically distributed with common mean and variance denoted $\E[K]$ and $\var(K)$, and $X_1, X_2, \cdots$ are identically distributed with common mean and variance denoted $\E[X]$ and $\var(X)$.
    \begin{enumerate}[label=(\alph*)]
        \item Derive formulas for $\E[N]$ and $\var(N)$ in terms of $\E[M]$, $\var(M)$, $\E[K]$, $\var(K)$.
        \item Derive formulas for $\E[Y]$ and $\var(Y)$ in terms of $\E[M]$, $\var(M)$, $\E[K]$, $\var(K)$, $\E[X]$, $\var(X)$.
        \item A crate contains $M$ cartons, where $M$ is geometrically distributed with parameter $p$. The $i$-th carton contains $K_i$ widgets, where $K_i$ is Poisson-distributed with parameter $\mu$. The weight of each widget is exponentially distributed with parameter $\lambda$. All these random variables are independent. Find the expected value and variance of the total weight of a crate.
    \end{enumerate}
\end{question}
\begin{answer} ~
    \begin{enumerate}[label=(\alph*)]
        \item According to the law of total expectation and total variance
        \begin{align}
            \E[N] &= \E[\E[N | M]] = \E[M \cdot \E[K]] = \E[M] \cdot \E[K] \\
            \var(N) &= \var(\E[N | M]) + \E[\var(N | M)] = \var(M) \cdot (\E[K])^2 + \E[M] \cdot \var(K)
        \end{align}
        \item \label{a:3.b} According to the law of total expectation and total variance
        \begin{align}
            &\E[Y] = \E[\E[Y | N]] = \E[N \cdot \E[X]] = \E[N] \cdot \E[X] = \E[M] \cdot \E[K] \cdot \E[X] \\
            &\begin{aligned}
                \var(Y) &= \var(\E[Y | N]) + \E[\var(Y | N)] = \var(N) \cdot (\E[X])^2 + \E[N] \cdot \var(X) \\ 
                &= \qty(\var(M) \cdot (\E[K])^2 + \E[M] \cdot \var(K)) \cdot (\E[X])^2 + \E[M] \cdot \E[K] \cdot \var(X) \\
            \end{aligned}
        \end{align}
        \item Let $Y$ be the total weight of a crate which satisfies
        \begin{equation}
            Y = \sum_{i=1}^M \sum_{j=1}^{K_i} W_{ij} \coloneqq \sum_{n=1}^N W_n
        \end{equation}
        where $N = \sum_{i=1}^M K_i$ is the total number of widgets in a crate and $\{W_n\}$ is the permutation of $\{W_{ij}\}$. So this questoin satisfies the condition of part \ref{a:3.b}. The expected value and variance of $Y$ can be computed as follows
        \begin{align}
            &\E[Y] = \E[M] \cdot \E[K] \cdot \E[W] = \frac{\mu}{\lambda p} \\
            &\begin{aligned}
                \var(Y) &= \qty(\var(M) \cdot (\E[K])^2 + \E[M] \cdot \var(K)) \cdot (\E[W])^2 + \E[M] \cdot \E[K] \cdot \var(W) \\ 
                &= \qty(\frac{(1 - p)\mu^2}{p^2} + \frac{\mu}{p}) \cdot \frac{1}{\lambda^2} + \frac{\mu}{\lambda^2 p}
            \end{aligned}
        \end{align}
    \end{enumerate}
    
\end{answer}

\begin{question}
    Find the PDF of the continuous random variable $X$  associated with the transform
    \begin{equation}
        M_X(s) = \frac{1}{3} \cdot \frac{2}{2 - s} + \frac{2}{3} \cdot \frac{3}{3 - s}
    \end{equation}
\end{question}
\begin{answer}
    Notice that the transform of an exponentially distributed random variable with parameter $\lambda$ is
    \begin{equation}
        M_X(s) = \frac{\lambda}{\lambda - s}
    \end{equation}
    So the PDF of $X$ is
    \begin{equation}
        f_X(x) = \frac{2}{3} \e^{-2x} + 2 \e^{-3x}, \quad x \geq 0
    \end{equation}
\end{answer}

\end{document}