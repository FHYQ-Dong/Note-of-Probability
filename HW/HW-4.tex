\documentclass[utf8]{article}
% \usepackage{ctex}
% \usepackage{circuitikz}
% \usepackage{tikz}
% \usepackage{graphicx}
\usepackage{hyperref}
\usepackage{geometry}
\usepackage{pdfpages}
\usepackage{booktabs}
\usepackage{subfigure}
\usepackage{float}
\usepackage{amsmath}
\usepackage{amssymb}
\usepackage{amsthm}
% math font
\usepackage{mathrsfs}
\usepackage{multirow}
\usepackage{inputenc}
\usepackage{fancyhdr}
\usepackage{enumitem}

\theoremstyle{definition}% default
\newtheorem{question}{Question} %
\theoremstyle{plain}% 
\newtheorem{answer}{Answer} %


\title{Problem Set 4}
\author{ FHYQ-Dong, Class **********, Student ID: ********** }
\date{\today}

\geometry{a4paper, scale=0.8}
\setlength{\columnsep}{20pt}
\pagestyle{fancy}
\fancyhf{}
\renewcommand{\headrulewidth}{0.5pt}
\renewcommand{\footrulewidth}{0.5pt}
\lhead{Problem Set 4}
\rhead{ FHYQ-Dong }
\cfoot{---~\thepage~---}

\hypersetup{
    colorlinks=true,
    linkcolor=black,
    filecolor=black,      
    urlcolor=black,
    citecolor=black,
}


\begin{document}

\maketitle
\thispagestyle{fancy}


\begin{question}
    An internet service provider uses 50 modems to serve the needs of 1000 customers. It is estimated that at a given time, each customer will need a connection with probability 0.01, independent of the other customers.
    \begin{enumerate}[label=(\alph*)]
        \item \label{q:1.a} What is the PMF of the number of modems in use at the given time?
        \item \label{q:1.b} Repeat part \ref{q:1.a} by approximating the PMF of the number of modems in use with a Poisson PMF.
        \item What is the probability that there are more customers needing a connection than there are modems? Provide an exact, as well as an approximate formula based on the Poisson approximation of part \ref{q:1.b}.
    \end{enumerate}
\end{question}
\begin{answer} ~
    \begin{enumerate}[label=(\alph*)]
        \item Let $X$ be the number of modems in use. Given that the number of modems is less than the the number of customers we have
        \begin{equation}
            p_X(x) = \left\{
            \begin{aligned}
                \binom{1000}{x} 0.01^x 0.99^{1000 - x}~~& \text{if } x = 0,1,2,\cdots,49 \\
                \sum_{k=50}^{1000} \binom{1000}{k} 0.01^k 0.99^{1000 - k}~~& \text{if } x = 50
            \end{aligned}\right.
        \end{equation}
        \item Let $\lambda = np = 1000 \times 0.01 = 10$, then the Poisson PMF is
        \begin{equation}
            p_X(x) = \left\{
            \begin{aligned}
                \frac{e^{-\lambda} \lambda^x}{x!}~~& \text{if } x = 0,1,2,\cdots,49 \\
                1 - \sum_{k=0}^{49} \frac{e^{-\lambda} \lambda^k}{k!}~~& \text{if } x = 50
            \end{aligned}\right.
        \end{equation}
        \item Exact formula:
        \begin{equation}
            p = \sum_{k=51}^{1000} \binom{1000}{k} (0.01)^k (0.99)^{1000 - k}
        \end{equation}
        Poisson approximation:
        \begin{equation}
            p = 1 - \sum_{k=0}^{50} \frac{e^{-\lambda} \lambda^k}{k!} = \sum_{k=51}^{1000} \frac{e^{-\lambda} \lambda^k}{k!}
        \end{equation}
    \end{enumerate}
\end{answer}


\begin{question}
    Let $X$ be a random variable with PMF
    \begin{equation}
        p_X(x) = 
        \begin{cases}
            x^2/a & \text{if } x = -3,-2,-1,0,1,2,3 \\
            0 & \text{otherwise}
        \end{cases}
    \end{equation}
    \begin{enumerate}[label=(\alph*)]
        \item Find $a$ and $\mathbf{E}[X]$.
        \item \label{q:2.b} What is the PMF of the random variable $Z = (X - \mathbf{E}[X])^2$?
        \item Using the result from part \ref{q:2.b}, find the variance of $X$.
        \item Find the varianceof $X$ using the formula $\text{var}(X) = \sum_x (x - \mathbf{E}[X])^2 p_X(x)$.
    \end{enumerate}
\end{question}
\begin{answer} ~
    \begin{enumerate}[label=(\alph*)]
        \item \begin{align}
            \sum_{x=-3}^{3} p_X(x) &= \frac{9+4+1+0+1+4+9}{a} = 28/a = 1 \Rightarrow a = 28 \\
            \mathbf{E}[X] &= \sum_{x=-3}^{3} x p_X(x) = \frac{1+4+9-9-4-1}{28} = 0
        \end{align}
        \item $\mathbf{E}[X] = 0 \Rightarrow Z = X^2$, given the domain and PMF of $X$ are both even:
        \begin{align}
            p_{Z}(z) = 
            \begin{cases}
                z/14 & \text{if } z = 0,1,4,9 \\
                0 & \text{otherwise}
            \end{cases}
        \end{align}
        \item \begin{align}
            \text{var}(X) = \mathbf{E}[Z] = \sum_{z=0}^{9} z p_{Z}(z) = \frac{1+4+9}{14} = 7
        \end{align}
        \item Given $\mathbf{E}[X] = 0$:
        \begin{align}
            \text{var}(X) = \sum_x (x - \mathbf{E}[X])^2 p_X(x) = \sum_{x=-3}^{3} x^2 p_X(x) = \frac{9+4+1+1+4+9}{28} = 7
        \end{align}
    \end{enumerate}
\end{answer}


\begin{question}
    Suppose that two teams play a series of games that ends when one of them has won 3 games. Suppose that each game played is, independently, won by team A with probability $p$. Find the expected number of games that are played. Also, show that this number is maximized when $p = 1/2$.
\end{question}
\begin{answer}
    Let $X$ be the number of games played, then $X \in \{3, 4, 5\}$.
    \begin{equation}
    \begin{aligned}
        p_{X}(3) &= p \cdot \binom{2}{2} p^2 + (1-p) \cdot \binom{2}{2} (1-p)^2 = p^3 + (1-p)^3 \\
        p_{X}(4) &= p \cdot \binom{3}{2} p^2 (1-p) + (1-p) \cdot \binom{3}{2} (1-p)^2 p = 3p^3(1-p) + 3p(1-p)^3 \\
        p_{X}(5) &= p \cdot \binom{4}{2} p^2 (1-p)^2 + (1-p) \cdot \binom{4}{2} (1-p)^2 p^2 = 6p^2(1-p)^2
    \end{aligned}
    \end{equation}
    So the expected number of games played is
    \begin{align}
        \mathbf{E}[X] = 6p^4 - 12p^3 + 3p^2 + 3p + 3
    \end{align}
    and its derivative and second derivative are
    \begin{equation}
    \begin{aligned}
        \frac{\mathrm{d}\mathbf{E}[X]}{\mathrm{d}p} &= 24p^3 - 36p^2 + 6p + 3 = 3(p-\frac{1}{2})(p-4-4\sqrt{2})(p-4+4\sqrt{2}) \\ 
        \frac{\mathrm{d}^2\mathbf{E}[X]}{\mathrm{d}p^2} &= 72p^2 - 72p + 6
    \end{aligned}        
    \end{equation}
    The derivative of $\mathbf{E}[X]$ only has one root $p = 1/2$ in $[0,1]$, and the second derivative is positive at that point. So the expected number of games played is maximized when $p = 1/2$.
\end{answer}


\begin{question}
    Let $X$ be a RV such that
    \begin{equation}
    \begin{cases}
        p_{X}(1) = p \\ 
        p_{X}(-1) = 1 - p
    \end{cases}
    \end{equation}
    Find $c \neq 1$, such that $\mathbf{E}[c^X] = 1$.
\end{question}
\begin{answer}
    Let $Z = c^X$, then the PMF of $Z$ is
    \begin{equation}
    \begin{cases}
        p_{Z}(c) = p_{X}(1) = p \\
        p_{Z}(1/c) = p_{X}(-1) = 1 - p
    \end{cases}
    \end{equation}
    The expectation of $Z$ is
    \begin{align}
        \mathbf{E}[Z] = p \cdot c + (1-p) \cdot \frac{1}{c} = \frac{c^2p - p + 1}{c}
    \end{align}
    Let $\mathbf{E}[Z] = 1$, then
    \begin{align}
        c^2p - p + 1 = c \Rightarrow c = \frac{1 - p}{p}~~(\text{c = 1 has been discarded})
    \end{align}
\end{answer}


\begin{question}
    Let $L$ be a geometric random variable with parameter $p = 0.02$. Find its median $k$, i.e., find $k$ that satisfies $\mathbf{P}(L \geq k) \geq 0.5$ and $\mathbf{P}(L \leq k) \geq 0.5$ simultaneously.
\end{question}
\begin{answer}
    \begin{align}
        p_{L}(l) &= (1 - p)^{l-1} p = 0.02 \times 0.98^{l-1} 
    \end{align}
    \begin{align}
    &\begin{aligned}
        \mathbf{P}(L \geq k) &= \sum_{l=k}^{\infty} p_{L}(l) = 0.02 \sum_{l=k}^{\infty} 0.98^{l-1} = 0.98^{k-1} \geq 0.5 \\
        &\Rightarrow k \leq 35
    \end{aligned} \\ 
    &\begin{aligned}
        \mathbf{P}(L \leq k) &= \sum_{l=1}^{k} p_{L}(l) = 0.02 \sum_{l=1}^{k} 0.98^{l-1} = 1 - 0.98^{k} \geq 0.5 \\
        &\Rightarrow k \geq 35
    \end{aligned}
    \end{align}
    Thus, the median of $L$ is 35.
\end{answer}

\end{document}