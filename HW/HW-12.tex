\documentclass[utf8]{article}
% \usepackage{ctex}
% \usepackage{circuitikz}
% \usepackage{tikz}
% \usepackage{graphicx}
\usepackage{hyperref}
\usepackage{geometry}
\usepackage{pdfpages}
\usepackage{booktabs}
\usepackage{subfigure}
\usepackage{float}
\usepackage{amsmath}
\usepackage{amssymb}
\usepackage{amsthm}
% math font
\usepackage{mathrsfs}
\usepackage{multirow}
\usepackage{multicol}
\usepackage{inputenc}
\usepackage{fancyhdr}
\usepackage{enumitem}
\usepackage{physics}
\usepackage{mathtools}

\theoremstyle{definition}% default
\newtheorem{question}{Question} %
\theoremstyle{plain}% 
\newtheorem{answer}{Answer} %
\newtheorem{theorem}{Theorem} % Define the theorem environment

\newcommand{\e}{\mathrm{e}}
\newcommand{\E}{\mathbf{E}}
\newcommand{\cov}{\mathrm{cov}}
\renewcommand{\var}{\mathrm{var}}


\title{Problem Set 12}
\author{ FHYQ-Dong, Class **********, Student ID: ********** }
\date{\today}

\geometry{a4paper, scale=0.8}
\setlength{\columnsep}{20pt}
\pagestyle{fancy}
\fancyhf{}
\renewcommand{\headrulewidth}{0.5pt}
\renewcommand{\footrulewidth}{0.5pt}
\lhead{Problem Set 12}
\rhead{ FHYQ-Dong }
\cfoot{---~\thepage~---}

\hypersetup{
  colorlinks=true,
  linkcolor=black,
  filecolor=black,   
  urlcolor=black,
  citecolor=black,
}


\begin{document}

\maketitle
\thispagestyle{fancy}

\begin{question}
    During each day, the probability that your computer's operating system crashes at least once is 5\%, independent of every other day. You are interested in the probability of at least 45 crash-free days out of the next 50 days.
    \begin{enumerate}[label=(\alph*)]
        \item \label{q:1.a} Find the probability of interest by using the normal approximation with 1/2 correction to the binomial.
        \item Repeat part \ref{q:1.a}, this time using the Poisson approximation to the binomial.
    \end{enumerate}
\end{question}
\begin{answer} ~
    \begin{enumerate}[label=(\alph*)]
        \item Let $X$ be the number of crash-free days in 50 days. Then $X$ follows a binomial distribution with parameters $n = 50$ and $p = 0.95$. By applying the normal approximation with 1/2 correction
        \begin{equation}
            \mathbf{P}(X \geq 45) = \mathbf{P}(44.5 \leq X) \approx 1 - \Phi\qty(\frac{44.5 - 50 \times 0.95}{\sqrt{50 \times 0.95 \times 0.05}}) \approx 1 - 0.02577 = 0.97423
        \end{equation}
        \item Notice that the number of crash days $Y$ also follows a binomial distribution with parameters $n = 50$ and $p = 0.05$. When $n$ is very large, we can use a Poisson RV with parameter $\lambda = np = 50 \times 0.05 = 2.5$ to approximate the distribution of $Y$. 
        \begin{equation}
            \mathbf{P}(X \geq 45) = \mathbf{P}(Y \leq 5) \approx \sum_{k=0}^{5} \e^{-\lambda} \frac{\lambda^k}{k!} \approx 0.962
        \end{equation}
    \end{enumerate}
\end{answer}

\begin{question}
    A factory produces $X_n$ gadgets on day $n$, where the $X_n$ are independent and identically distributed random variables, with mean 5 and variance 9.
    \begin{enumerate}[label=(\alph*)]
        \item Find an approximation with 1/2 correction to the probability that the total number of gadgets produced in 100 days is less than 440.
        \item  Find (approximately) the largest value of $n$ such that
        \begin{equation}
            \mathbf{P}(X_1 + X_2 + \cdots + X_n \geq 200 + 5n) \leq 0.05
        \end{equation}
        \item Let $N$ be the first day on which the total number of gadgets produced exceeds 1000. Calculate an approximation to the probability that $N \geq 220$.
    \end{enumerate}
\end{question}
\begin{answer} ~
    \begin{enumerate}[label=(\alph*)]
        \item Let $S = X_1 + \cdots X_100$ be the total number of gadgets produced in 100 days, then according to the CLT
        \begin{equation}
            \mathbf{P}(S < 440) = \mathbf{P}(S \leq 439.5) \quad\Leftrightarrow\quad \mathbf{P}\qty(\frac{S - 5 \times 100}{\sqrt{100} \times 3} < \frac{439.5 - 5 \times 100}{\sqrt{100} \times 3}) \approx \Phi\qty(- \frac{60.5}{30}) \approx 0.02186
        \end{equation}
        \item Notice that $\E[X_i] = 5$ and $\var(X_i) = 9$, then
        \begin{equation}
            \mathbf{P}(X_1 + \cdots + X_n \geq 200 + 5n) = \mathbf{P}\qty(\frac{X_1 + \cdots + X_n - 5n}{3\sqrt{n}} \geq \frac{200}{3\sqrt{n}}) \leq 0.05 \quad\Rightarrow\quad 1 - \Phi\qty(\frac{200}{3\sqrt{n}}) \leq 0.05
        \end{equation}
        which yields (by checking the standard normal table)
        \begin{equation}
            \frac{200}{3\sqrt{n}} \geq 1.65 \quad\Rightarrow\quad n \leq 1632
        \end{equation}
        \item Let $S_n = X_1 + \cdots + X_n$ be the total number of gadgets produced in $n$ days, then
        \begin{equation}
            \mathbf{P}(N \leq 220) = \mathbf{P}(S_{219} \leq 1000) = \mathbf{P}\qty(\frac{S_{219} - 5 \times 219}{3\sqrt{219}} \leq \frac{1000 - 5 \times 219}{3\sqrt{219}}) = \Phi(-2.1398) \approx 0.01618
        \end{equation}
    \end{enumerate}
\end{answer}

\begin{question}
    Let $X_1, Y_1, X_2, Y_2, \ldots$ be independent random variables, uniformly distributed in the unit interval $[0, 1]$, and let
    \begin{equation}
        W = \frac{(X_1 + X_2 + \cdot + X_{16}) - (Y_1 + Y_2 + \cdot + Y_{16})}{16}
    \end{equation}
    Find a numerical approximation to the quantity
    \begin{equation}
        \mathbf{P}(\abs{W - \E[W]} < 0.001)
    \end{equation}
\end{question}
\begin{answer}
    Let $Z_i = X_i - Y_i$, then $\E[Z_i] = 0$ and $\var(Z_i) = 1/6$. Then $W = (Z_1 + \cdots Z_{16}) / 16$, whose $\E[W] = 0$ and $\var(W) = 1/96$. Therefore, according to CLT
    \begin{equation}
        \mathbf{P}(\abs{W - \E[W]} < 0.001) = \mathbf{P}(\frac{\abs{W}}{\sqrt{1/96}} < \frac{0.001}{\sqrt{1/96}}) \approx \Phi(0.001\sqrt{96}) - \Phi(-0.001\sqrt{96}) \approx 0.00782
    \end{equation}
\end{answer}

\begin{question}
    It is well known that infants born to mothers who smoke tend to be small and prone to a range of ailments. It is conjectured that also they look abnormal. Nurses were shown selections of photographs of babies, one half of whom had smokers as mothers; the nurses were asked to judge from a baby's appearance whether or not the mother smoked. In 1500 trials the correct answer was given 910 times. Is the conjecture plausible? If so, why?
\end{question}
\begin{answer}
    Proof by contradiction. If the conjecture is not plausible, then the number correct answers $X$ should follows a binomial distribution with parameters $n = 1500$ and $p = 0.5$. According to the CLT
    \begin{equation}
        \mathbf{P}\qty(\frac{X - 1500 \times 0.5}{\sqrt{1500 \times 0.5 \times 0.5}} \geq x) = \mathbf{P}\qty(\frac{X - 750}{\sqrt{375}} \geq x) \approx 1 - \Phi(x)
    \end{equation}
    Now $X = 910$, then
    \begin{equation}
        \frac{X - 750}{\sqrt{375}} = \frac{910 - 750}{\sqrt{375}} \approx 8.26 \geq 8 \quad\Rightarrow\quad \mathbf{P}\qty(\frac{X - 750}{\sqrt{375}}) \leq 1 - \Phi(8) \approx 6.6613 \times 10^{-16}
    \end{equation}
    Since the probability of getting 910 correct answers is extremely small, we can conclude that the conjecture is plausible.
\end{answer}

\begin{question}
    \textbf{Conputional Problem.} In this problem, we want to demonstrate the central limit theorem by writing a computer program. Let $X_1, \ldots, X_n$ be i.i.d. random variables following the uniform distribution within the interval $[0, 1]$. Let 
    \begin{equation}
        Z_n = \frac{X_1 + \cdots + X_n - n\mu}{\sigma\sqrt{n}}
    \end{equation}
    where $\mu$ and $\sigma$ denote the mean and the standard deviation of $X_i$, respectively. Write a computer program to generate 100000 samples of $Z_n$ and draw the histogram of the samples for $n = 1, n = 2, n = 5$ and $n = 50$ respectively. \\ 
    You don't have to upload your programs. Just take a screenshot of your programs as well as the results, and show them in your homework. There is no restriction on the programming language you use.
\end{question}
\begin{answer} ~
    \begin{figure}[H]
        \centering
        \includegraphics[width=0.95\linewidth]{images/HW-12.5-result.png}
        \caption{Result of the computational problem.}
    \end{figure}
    \begin{figure}[H]
        \centering
        \includegraphics[width=0.95\linewidth]{images/HW-12.5-program.png}
        \caption{Program of the computational problem.}
    \end{figure}
\end{answer}

\end{document}
